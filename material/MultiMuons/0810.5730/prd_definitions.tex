%
% Input some definitions
%
%
%----------  PHYSICS COMMANDS
%
\def \Et {{\rm E}_{\rm T}}
\def \Pt {{\rm P}_{\rm T}}
\def \Pz {{\rm P}_{\rm Z}}
\def \enu {\epsilon_{\nu}}
\def \stw {$\sin^{2}\theta_{W}$}
\newcommand{\MET}{\mbox{$\protect \raisebox{.3ex}{$\not$}\et$}}
\newcommand{\METC}{\mbox{$\protect \raisebox{.3ex}{$\not$}\etc$}}
%\def \MET {\not\!\Et}
\def\deg{^\circ}
\def\qbar{{\bar q}}
\def\nubar{{\bar \nu}}
\def\W{{\em W\/ }}
\def\Z0{${\em Z^0\/}$}
\def \lum {{\cal L}}
\def\epem{{\rm e^{+}e^{-}}}
\def\tptm{{\tau^{+}\tau^{-}}}
\def\roots{${\sqrt s}\:$}
\def\r#1 {$^{#1}$}
\def\sigW {$\sigma\cdot$B(\W$\rightarrow~$e $\nu$) }
\def\sigZ {$\sigma\cdot$B(\Z0$\rightarrow~\epem$) }
\hyphenation{brem-sstrah-lung proc-ess}
%
\def\first{{\mbox{$I\leq$}4~GeV}}
\def\second{{QC=\mbox{$I\leq$}~4~GeV+\mbox{$s_{ip}\leq$}~4}}
\def\third{{QC+\mbox{$\delta \phi \geq 2$}}}
\def\fourth{{QC+\mbox{$|\cos \theta^{*}| \leq 0.4$}}}
\def\fifth{{QC+\mbox{$\delta \phi \geq 2$}+\mbox{$|\cos \theta^{*}|\leq 0.4$}}} 
\def\six{{QC+\mbox{$\sum p_t  \leq$}40~GeV+\mbox{$\sum s_{ip} \leq 30$}}}
\def\seven{{\mbox{$\delta \phi \geq 2$}}}
\def\eight{{\mbox{$|\cos \theta^{*}| \leq 0.4$}}}
\def\nine{{\mbox{$\delta \phi \geq 2$}+\mbox{$|\cos \theta^{*}| \leq 0.4$}}} 
%
%\input moredefs.tex
%
%
%
%
\newcommand{\etc}{{\rm E}_{\scriptscriptstyle\rm T}^{\scriptscriptstyle\rm C}}
\newcommand{\et}{{\rm E}_{\scriptscriptstyle\rm T}}
\newcommand{\etcone}{{\rm E}_{\scriptscriptstyle\rm T}^{cone}}
\newcommand{\abseta}{\mid \eta^{det} \mid \leq}
\newcommand{\abz}{\mid z \mid \leq}
\newcommand{\fb}{f_{b}}
\newcommand{\ks}{K_{s}^{0}}
\newcommand{\pich}{\pi^{\pm}}
\newcommand{\piz}{ \pi^{0} }
\newcommand{\bigz}{{\cal Z}}
\newcommand{\emf}{f_{em}}
\newcommand{\deltar}{\sqrt{\Delta \eta ^{2}+ \Delta \phi ^{2}}}
\newcommand{\etprime}{{\rm E}_{\scriptscriptstyle\rm T'}}
\newcommand{\ptran}{{\rm P}_{\scriptscriptstyle\rm T}}
\newcommand{\met}{\mbox{$\protect \raisebox{.3ex}{$\not$}\et \ $}}
\newcommand{\wenu}{W \rightarrow e \nu}
\newcommand{\wmunu}{W \rightarrow \mu \nu}
\newcommand{\wlep}{W \rightarrow \rm{lepton}\, \nu}
\newcommand{\zv}{{\rm z}_{vertex}}
\newcommand{\wbb} {W b\bar{b} }
\newcommand{\wcc} {W c\bar{c} }
\newcommand{\ppbar}{p\bar{p}}
\newcommand{\qqbar}{q\bar{q}}
\newcommand{\ttbar}{t\bar{t}}
\newcommand{\bbbar}{b\bar{b}}
\newcommand{\ccbar}{c\bar{c}}
\newcommand{\ppbb} { \ppbar \rightarrow  \bbbar }
%\newcommand{\zee}{Z \rightarrow e^{+}e^{-} }
\newcommand{\bele}{b \rightarrow c e \nu_{e} }         
\newcommand{\blnu}{b \rightarrow c l \nu_{l} }         
\newcommand{\mtran}{{\rm M}_{\scriptscriptstyle\rm T}}
\newcommand{\acceff}{\rm{A} \times \epsilon}
\newcommand{\bbar} {\bar{b}}   
\newcommand{\gbb} { g \rightarrow b\bar{b} }
\newcommand{\gcc} { g \rightarrow c\bar{c}}
\newcommand{\tbar} { \bar{{t}}  }                                
\newcommand{\Lik}{\mbox{$\mathcal{L}$ }}
\newcommand{\Ki}{\mbox{$\chi^{2}$ }}
\newcommand{\mPr}{\mathcal{P}}
\newcommand{\CPr}{\mbox{$\mathcal{C}$ }}
\newcommand{\mLik}{\mathcal{L}}
\newcommand{\mCPr}{\mathcal{C}}
% dilepton symbols1
\def \mc {\multicolumn}
\def \pb    {pb$^{-1} $}
\def \DeltaPhi {$\Delta \phi_{\ell\,\ell \ }$} 
\def \mtop {$M_{top} \ $}
\def \mtopev {$M_{top}^{event} $}
\def \mw {$M_{W} \ $}
\def \ztau   {$Z\rightarrow\tau\tau \:$}
\def \DeltaPhil {$\Delta \phi{(\MET,\ell) \ }$} 
\def \DeltaPhij {$\Delta \phi{(\MET,j) \ }$} 
\def \TTbar {$t\overline{t} \; \;$}
\def \dpemu {\Delta \phi_{e\mu}}
\def \Mt {M_{top}}
\def \mtenu  {M_{T}^{e\nu}}
\def \lum {\cal L}
\def \intlum {\int {\cal L} dt}
\def \Zee {Z^{0} \rightarrow e^{+}e^{-}}
\def \Zmumu {Z^{0} \rightarrow \mu^{+}\mu^{-}}
\def \emu {e \mu}  
\def \temux {\ttbar \rightarrow \emu + X}
\def \tljx {\ttbar \rightarrow \ell \nu q \bar{q}^{\prime} b \bar{b} X}
\def \tllx {\ttbar \rightarrow \ell^+ \bar{\nu} \ell^- \nu b \bar{b} X}
\def \thad {\ttbar \rightarrow q \bar{q} b q \bar{q} \bar{b} X}   
\def \Ete {E_T^{e}}
\def \Ptmin {P_T^{min}}
\def \Ptmu {P_T^{\mu}}
\def \Etmiss {{\not}{E_T}}
% end of dilepton
%---------- UNITS, SYMBOLS
%
\newcommand{\imb}{ \mu {\rm b}^{-1} }
\newcommand{\inb}{ {\rm nb}^{-1} }
\newcommand{\ipb}{ {\rm pb}^{-1} }
\newcommand{\degs}{\mbox{$^{\circ}$}}
\newcommand{\gsim}{\mbox{\small$\stackrel{>}{\sim}$\normalsize}}
\newcommand{\lsim}{\mbox{\small$\stackrel{<}{\sim}$\normalsize}}
\newcommand{\gev}  { {\rm GeV}}
\newcommand{\tev}  { {\rm TeV}}
\newcommand{\gevc} { {\rm GeV/}c}
\newcommand{\gevcc}{ {\rm GeV/}c^2}

%
%---------- TYPE SETTING
%
\newcommand{\etal}{{\em et al.}}
\newcommand{\tableskip}{\vskip 5pt plus3pt minus1pt \relax}
\newcommand{\tindent}{\hskip 17pt}
\newcommand{\hfull}{\hspace*{\fill}}
\newcommand{\tline}{\protect\linebreak[4]\hfull}
\newcommand{\linespace}[1]{\protect\renewcommand{\baselinestretch}{#1}
  \footnotesize\normalsize}
%
%---------- Journal names
%
%\newcommand{\prl}[1]{Phys. Rev. Lett {\bf #1}}
%\newcommand{\prev}[1]{Phys. Rev. {\bf #1}}
%\newcommand{\prd}[1]{Phys. Rev. D {\bf #1}}
%\newcommand{\zs}[1]{Z. Phys. {\bf #1}}
%\newcommand{\ncim}[1]{Nuovo Cim. {\bf #1}}
%\newcommand{\plet}[1]{Phys. Lett. {\bf #1}}
%\newcommand{\prep}[1]{Phys. Rep. {\bf #1}}
%\newcommand{\rmp}[1]{Rev. Mod. Phys. {\bf #1}}
%\newcommand{\nphy}[1]{Nucl. Phys. {\bf #1}}
%\newcommand{\nim}[1]{Nucl. Instrumen. Meth. {\bf #1}}
%

%------------- Figure commands and macros
%
%
%  Called the same way epsffile is called.  Difference is it will center
%  the graphic in the page useing the center environment.
%
\def\gepsfcentered#1{
  \def\testit{#1}
  \def\lbracket{[}
  \ifx\testit\lbracket
    \let\dofilecmd=\gepsfwithopt
  \else
    \let\dofilecmd=\gepsfnoopt
  \fi
  \dofilecmd}

\def\gepsfnoopt#1{
  \begin{center}
  \leavevmode
  \epsffile{#1}
  \end{center}}

\def\gepsfwithopt#1 #2 #3 #4]#5{
  \begin{center}
  \leavevmode
  \gepsfmaxx=0.94\textwidth
  \epsffile[#1 #2 #3 #4]{#5}
  \end{center}}

%
%  Auto sizing for epsf figures that are larger than the text width.
%
\newdimen\gepsfmaxx
\gepsfmaxx=0.94\textwidth
\def\epsfsize#1#2{
  \ifnum \epsfxsize=0
    \ifnum \epsfysize=0
      \ifnum #1 > \gepsfmaxx
        \gepsfmaxx
	%\message{Did scaling.}
      \else
        #1
	%\messaeg{Used nat scaling}
      \fi
    \else
      \epsfxsize
      %\message{Using what ever.}
    \fi
  \else
    \epsfxsize
    %\message{Again, using whatever.}
  \fi
  %\message{Hi epsfxsize is \the\epsfxsize ...}
  %\message{epsfysize is \the\epsfysize ...}
  %\message{Hi first arg is \the#1 ...}
  %\message{Second arg is \the#2 ...}
}
