%=============================================================================
%%% General Definitions
%=============================================================================
%%% My details
\newcommand{\MyInstitute}{University Of Cyprus\xspace}
\newcommand{\MyInstituteShort}{UCY\xspace}
\newcommand{\MyName}{Alexandros Attikis\xspace}
\newcommand{\MyNameShort}{A. Attikis\xspace}

% Implicit math mode columns
\newcolumntype{L}{>{$}l<{$}}
\newcolumntype{C}{>{$}c<{$}}
\newcolumntype{R}{>{$}r<{$}}
%\newdateformat{mydate}{\THEDAY\xspace \monthname[\THEMONTH]\xspace \THEYEAR}

%=============================================================================
%%% Environments
%=============================================================================
\newenvironment{TypewriterFont}{\ttfamily}{\par} %\texttt{}

%%% Re-Define name for Table Of Contents (default is "Contents")
\renewcommand{\contentsname}{Table of Contents}

%%% Define a "make list of acronyms" command, similar to \listoffigures. \listoftables etc..
\newcommand{\listofacronymsname}{List of Acronyms}{}
\newcommand{\listofacronyms}{%
  \chapter*{\listofacronymsname}%
  \addcontentsline{toc}{chapter}{\listofacronymsname}%
  \label{sec:acronyms}%
}

%%% Define new environment where the margin values are customarily redefined
\newenvironment{changemargin}[3]{
  \begin{list}{}{
      %\setlength{\topsep}{#1}
      \setlength{\topmargin}{#1}
      \setlength{\leftmargin}{#2}
      \setlength{\rightmargin}{#3}
      \setlength{\listparindent}{\parindent}
      \setlength{\itemindent}{\parindent}
      \setlength{\parsep}{\parskip}
    }
  \item[]}{\end{list}}

%=============================================================================
%%% Reference and Citations Shortcuts
%=============================================================================
% Reference
\newcommand{\refchap}[1]{Chapter~\ref{chap:#1}\xspace}
\newcommand{\refsec}[1]{Section~\ref{sec:#1}\xspace}
\newcommand{\refdisec}[2]{Sections~\ref{sec:#1} and~\ref{sec:#2}\xspace}
\newcommand{\refapp}[1]{Appendix~\ref{app:#1}\xspace}
\newcommand{\refdiapp}[2]{Appendices~\ref{app:#1} and~\ref{app:#2}\xspace}
\newcommand{\reffig}[1]{Fig.~\ref{fig:#1}\xspace}
\newcommand{\reffigplural}{Figures}
\newcommand{\refdifig}[2]{\reffigplural~\ref{fig:#1} and~\ref{fig:#2}\xspace}
\newcommand{\reftrifig}[3]{\reffigplural~\ref{fig:#1}, \ref{fig:#2}, and~\ref{fig:#3}\xspace}
\newcommand{\reffigbegin}[1]{Fig.~\ref{fig:#1}\xspace}
\newcommand{\refdifigbegin}[2]{Figs.~\ref{fig:#1} and~\ref{fig:#2}\xspace}
\newcommand{\reftrifigbegin}[3]{Figs.~\ref{fig:#1}, \ref{fig:#2}, and~\ref{fig:#3}\xspace}
\newcommand{\refsixfigbegin}[6]{Figs.~\ref{fig:#1}, \ref{fig:#2}, \ref{fig:#3}, \ref{fig:#4}, \ref{fig:#5}, and~\ref{fig:#6}\xspace}
\newcommand{\refsevenfigbegin}[7]{Figs.~\ref{fig:#1}, \ref{fig:#2}, \ref{fig:#3}, \ref{fig:#4}, \ref{fig:#5}, \ref{fig:#6}, and~\ref{fig:#7}\xspace}
\newcommand{\reftab}[1]{Table~\ref{tab:#1}\xspace}
\newcommand{\refditab}[2]{Tables~\ref{tab:#1} and~\ref{tab:#2}\xspace}
\newcommand{\refcite}[1]{Ref.~\cite{#1}\xspace}
\newcommand{\refeq}[1]{Eq.~\eqref{eq:#1}}
\newcommand{\refdieq}[2]{Eq.~\eqref{eq:#1} and \eqref{eq:#2}\xspace}
\newcommand{\reftrieq}[3]{Eq.~\eqref{eq:#1}, \eqref{eq:#2} and \eqref{eq:#3}\xspace}
\newcommand{\refeqrange}[2]{Eqs.~\eqref{eq:#1}-\eqref{eq:#2}}
\newcommand{\reffigrange}[2]{Figs.~\ref{fig:#1}-\ref{fig:#2}}
%%% Citations
\newcommand{\refdicite}[2]{Refs.~\cite{#1} and \cite{#2}\xspace}
\newcommand{\reftricite}[3]{Refs.~\cite{#1},~\cite{#2} and \cite{#3}\xspace}
\newcommand{\refquadcite}[4]{Refs.~\cite{#1},~\cite{#2},~\cite{#3} and \cite{#4}\xspace}
\newcommand{\reftrisec}[3]{Sections~\ref{sec:#1},~\ref{sec:#2} and ~\ref{sec:#3}\xspace}
\newcommand{\refquadsec}[4]{Sections~\ref{sec:#1},~\ref{sec:#2},~\ref{sec:#3} and ~\ref{sec:#4}\xspace}
\newcommand{\refsecAndPage}[1]{Section~\ref{sec:#1} on page \pageref{sec:#1}\xspace}
\newcommand{\refappAndPage}[1]{Appendix~\ref{app:#1} on page \pageref{app:#1}\xspace} 
\newcommand{\reffigAndPage}[1]{Fig.~\ref{fig:#1} on page \pageref{fig:#1}\xspace}
\newcommand{\reftabAndPage}[1]{Table~\ref{tab:#1} on page \pageref{tab:#1}\xspace}
\newcommand{\reftritab}[3]{Tables~\ref{tab:#1},~\ref{tab:#2} and~\ref{tab:#3}\xspace}
\newcommand{\refciteAndPage}[1]{Ref.~\cite{#1} on page \pageref{#1} \xspace}
\newcommand{\refeqAndPage}[1]{Eq.~\eqref{eq:#1} on page \pageref{eq:#1} \xspace}

%=============================================================================
%%% Acronyms
%=============================================================================
\newcommand{\ATLAS}{\ac{ATLAS}\xspace}
\newcommand{\IEF}{\ac{IEF}\xspace}

%============================================================================= 
%%% General symbols
%============================================================================= 
\makeatletter
\newcommand{\romanNumber}[1]{\fontfamily{roman}\selectfont\romannumeral#1\normalfont}
\newcommand{\RomanNumber}[1]{\fontfamily{roman}\selectfont\expandafter\@slowromancap\romannumeral#1@\normalfont}
\newcommand{\Tan}[1]{\ensuremath{\tan\left(#1\right)}\xspace}
\newcommand{\Cos}[1]{\ensuremath{\tan\left(#1\right)}\xspace}
\newcommand{\Sin}[1]{\ensuremath{\tan\left(#1\right)}\xspace}
\makeatother
\newcommand{\sOrder}[1]{\ensuremath{\mathcal{O}\left(#1\right)}\xspace}
\newcommand{\sOrderOf}[1]{\sOrder{#1}}
\newcommand{\abs}[1]{\absoluteValue{#1}\xspace}
\newcommand{\sign}[1]{\ensuremath{\mathrm{sgn}\left(#1\right)}\xspace}
\newcommand{\absoluteValue}[1]{\ensuremath{\lvert#1\rvert}\xspace}
\newcommand{\Units}[1]{\tag{\ensuremath{#1}} }
\newcommand{\BR}[1]{\ensuremath{\mathcal{B}\left(#1\right)}\xspace}
\newcommand{\CTau}[1]{\ensuremath{c\tau}_{#1}\xspace}
\newcommand{\XSection}{\ensuremath{\sigma}\xspace}
\newcommand{\error}[1]{\ensuremath{\sigma_{#1}}\xspace}
\newcommand{\errorPow}[2]{\ensuremath{\sigma_{#1}^{#2}}\xspace}

%%% This Ensures that two itemize environments side by side in a
%%% two-column environment are actaully top-aligned!
%%% See: http://tex.stackexchange.com/questions/22916/top-alignment-of-itemize-in-columns-of-beamer
\makeatletter
\define@key{beamerframe}{t}[true]{% top
  \beamer@frametopskip=.2cm plus .5\paperheight\relax%
  \beamer@framebottomskip=0pt plus 1fill\relax%
  \beamer@frametopskipautobreak=\beamer@frametopskip\relax%
  \beamer@framebottomskipautobreak=\beamer@framebottomskip\relax%
  \def\beamer@initfirstlineunskip{}%
}
\makeatother
%=============================================================================
%%% Kinematic Variables
%=============================================================================
\newcommand{\pT}{\ensuremath{p_{\mathrm{T}}}\xspace}
\newcommand{\pZ}{\ensuremath{p_{\mathrm{z}}}\xspace}
\newcommand{\eT}{\ensuremath{E_{\mathrm{T}}}\xspace}
\newcommand{\deltaR}{\ensuremath{\Delta \mathrm{R}}\xspace}
\newcommand{\deltaPhi}{\ensuremath{\Delta \phi}\xspace}
\newcommand{\eTVis}{\ensuremath{E_{\mathrm{T}}^{\mathrm{vis}}}\xspace}
\newcommand{\eTCorr}{\ensuremath{E_{\mathrm{T,~corr}}}\xspace}
\newcommand{\eTSub}[1]{\ensuremath{E_{\mathrm{T,~#1}}}\xspace}
\newcommand{\etaVis}{\ensuremath{\eta^{\mathrm{vis}}}\xspace}
\newcommand{\etaCorr}{\ensuremath{\eta_{\mathrm{corr}}}\xspace}
\newcommand{\etaSub}[1]{\ensuremath{\eta_{\mathrm{#1}}}\xspace}
\newcommand{\thetaCorr}{\ensuremath{\theta_{\mathrm{corr}}}\xspace}
\newcommand{\pTPow}[1]{\ensuremath{p_{\mathrm{T}}^{\mathrm{#1}}}\xspace}
\newcommand{\pZPow}[1]{\ensuremath{p_{\mathrm{z}}^{\mathrm{#1}}}\xspace}
\newcommand{\pTSub}[1]{\ensuremath{p_{\mathrm{T},~\mathrm{#1}}}\xspace}

%=============================================================================
%%% TTI WG (HLTaus)
%=============================================================================
\newcommand{\Pythia}{\text{PYTHIA}{}6\xspace}
\newcommand{\Tauola}{\text{TAUOLA}\xspace}
\newcommand{\Geant}{\text{GEANT}{}4\xspace}
\newcommand{\Madgraph}{\text{MadGraph}\xspace}
\newcommand{\Powheg}{\text{POWHEG}\xspace}
%
\newcommand{\MCTau}{\ensuremath{\mathrm{\tau}_{\mathrm{mc}}}\xspace}
\newcommand{\McVtx}{\ensuremath{\text{MC Vertex}}\xspace}
\newcommand{\MCTauVis}{\ensuremath{\mathrm{\tau}_{\mathrm{mc}}^{\mathrm{vis}}}\xspace}
\newcommand{\LOneTrigger}{\ensuremath{\text{L1 Trigger }}\xspace}
\newcommand{\LOneTauTrigger}{\ensuremath{\text{L1 Tau Trigger }}\xspace}
\newcommand{\LOneVtx}{\ensuremath{\text{L1 Vertex}}\xspace}
\newcommand{\LOneTk}{\ensuremath{\text{L1 Track}}\xspace}
\newcommand{\LOneTks}{\ensuremath{\text{L1 Tracks}}\xspace}
\newcommand{\LOnePixTk}{\ensuremath{\text{L1 Pixel Track}}\xspace}
\newcommand{\LOnePixTks}{\ensuremath{\text{L1 Pixel Tracks}}\xspace}
%\newcommand{\LOneTkTau}{\ensuremath{\text{L1 } \mathrm{\tau}_{\mathrm{tk}}}\xspace}
%\newcommand{\LOneTkTaus}{\ensuremath{\text{L1 } \mathrm{\tau}_{\mathrm{tk}}}'s\xspace}
\newcommand{\LOneTkTau}{\LOneTk--Tau\xspace}
\newcommand{\LOneTkTaus}{\LOneTk--Taus\xspace}
\newcommand{\LOneTkCluster}{\ensuremath{\text{L1 Tk Cluster}}\xspace}
\newcommand{\LOneTkClusters}{\ensuremath{\text{L1 Tk Clusters}}\xspace}
\newcommand{\LOneLdgTk}{\ensuremath{\text{L1 Tk}^{\mathrm{ldg}}}\xspace}
\newcommand{\LOneTauLdgTk}{\ensuremath{\text{L1 } \mathrm{\tau}_{\mathrm{tk}}^{\mathrm{ldg}}}\xspace}
\newcommand{\LOneCaloTau}{\ensuremath{\text{L1 Calo Tau}}\xspace}
\newcommand{\LOneCaloTaus}{\ensuremath{\text{L1 Calos Taus}}\xspace}
\newcommand{\LOneCaloTauCandidates}{\ensuremath{\mathrm{L1}} \ensuremath{\mathrm{\tau}_{\mathrm{calo}}} candidates\xspace}
%
\newcommand{\ldgtk}[1]{\ensuremath{\mathrm{ldg~\tk{{#1}}}}\xspace}
\newcommand{\Weight}[1]{\ensuremath{\mathcal{W}_{\mathrm{#1}}}\xspace}
\newcommand{\weight}[1]{\ensuremath{\mathrm{w_{\mathrm{#1}}}}\xspace}
\newcommand{\tk}[1]{\ensuremath{\mathrm{tk_{\mathrm{#1}}}}\xspace}
\newcommand{\cluster}{\ensuremath{\mathrm{cluster}}\xspace}
\newcommand{\invMass}[1]{\ensuremath{\mathrm{m\left(#1\right)}}\xspace}
\newcommand{\calo}{\ensuremath{\mathrm{calo}}\xspace}
\newcommand{\xPOCA}[1]{\ensuremath{\mathrm{x}_{\mathrm{0}}^{\mathrm{#1}}}\xspace}
\newcommand{\yPOCA}[1]{\ensuremath{\mathrm{y}_{\mathrm{0}}^{\mathrm{#1}}}\xspace}
\newcommand{\zPOCA}[1]{\ensuremath{\mathrm{z}_{\mathrm{0}}^{\mathrm{#1}}}\xspace}
\newcommand{\rPOCAVec}[1]{\ensuremath{\vec{\mathrm{r}}_{\mathrm{0}}^{\mathrm{#1}}}\xspace}
\newcommand{\dXY}[1]{\ensuremath{\mathrm{d}_{\mathrm{xy}}^{\mathrm{#1}}}\xspace}
\newcommand{\dZero}[1]{\ensuremath{\mathrm{d}_{\mathrm{0}}^{\mathrm{#1}}}\xspace}
\newcommand{\dZeroVec}[1]{\ensuremath{\vec{\mathrm{d}}_{\mathrm{0}}^{\mathrm{#1}}}\xspace}
\newcommand{\dZeroAbs}[1]{\ensuremath{\abs{\mathrm{d}_{\mathrm{0}}^{\mathrm{#1}}}}\xspace}
\newcommand{\thetaZero}[1]{\ensuremath{\mathrm{\theta}_{\mathrm{0}}^{\mathrm{#1}}}\xspace}
\newcommand{\phiZero}[1]{\ensuremath{\mathrm{\phi}_{\mathrm{0}}^{\mathrm{#1}}}\xspace}
\newcommand{\zPOCASub}[1]{\ensuremath{\mathrm{z}_{\mathrm{0,~#1}}}\xspace}
\newcommand{\zVtx}[1]{\ensuremath{\mathrm{z}_{\mathrm{vtx}}^{\mathrm{#1}}}\xspace}
\newcommand{\xPV}[1]{\ensuremath{\mathrm{x}_{\mathrm{PV}}^{\mathrm{#1}}}\xspace}
\newcommand{\yPV}[1]{\ensuremath{\mathrm{y}_{\mathrm{PV}}^{\mathrm{#1}}}\xspace}
\newcommand{\zPV}[1]{\ensuremath{\mathrm{z}_{\mathrm{PV}}^{\mathrm{#1}}}\xspace}
%\newcommand{\redChiSq}[1]{\ensuremath{\chi^{2}_{\mathrm{#1}}/\mathrm{d.o.f}}\xspace}
\newcommand{\redChiSq}[1]{\ensuremath{\chi^{2}_{\mathrm{N}{\mathrm{#1}}}}\xspace}
\newcommand{\chiSq}{\ensuremath{\chi^{2}}\xspace}
\newcommand{\NStubs}{\ensuremath{N_{\mathrm{stubs}}}\xspace}
\newcommand{\NPsStubs}{\ensuremath{N_{\mathrm{stubs}}^{\mathrm{PS}}}\xspace}
\newcommand{\SumOver}[2]{\ensuremath{\sum\limits_{\mathrm{#1}}^{\mathrm{#2}}} \xspace}
\newcommand{\RelIsoEqn}{\ensuremath{\SumOver{i}{iso-tks}\left(\frac{\mathrm{p}_{\mathrm{T}}^{\mathrm{i}}}{\mathrm{p}_{\mathrm{T}}^{\mathrm{ldg}}}\right)}\xspace}
%\newcommand{\VtxIsoEqn}{\ensuremath{\SumOver{i}{tracks} \left( \abs{\zPOCA{\tk{m}} - \zPOCA{\tk{i}}} \leq X \sCentiMeter \right) = 0}\xspace}
\newcommand{\VtxIsoEqn}{\ensuremath{\mathrm{N}_{\mathrm{iso-tks}}\left( \abs{\zPOCA{\tk{m}} - \zPOCA{\tk{i}}} \leq X \sCentiMeter \right) = 0}\xspace}
\newcommand{\VtxIsoEqnAlt}[1]{min\ensuremath{\left(\abs{\zPOCA{\tk{m}} - \zPOCA{\tk{i}}}\right) \geq #1 \sCentiMeter}\xspace}
\newcommand{\TrigProb}{\ensuremath{\mathrm{p}_{\tau}}\xspace}
\newcommand{\TrigProbPow}[1]{\ensuremath{\mathrm{p}_{\tau}^{#1}}\xspace}
\newcommand{\TrigDiProb}{\ensuremath{\mathrm{p}_{\DiTau}}\xspace}
\newcommand{\TrigDiProbSub}[1]{\ensuremath{\mathrm{p}_{\DiTau,~#1}}\xspace}
\newcommand{\TrigEff}{\ensuremath{\varepsilon_{\tau}}\xspace}
\newcommand{\TrigEffSub}[1]{\ensuremath{\varepsilon_{\tau,~#1}}\xspace}
\newcommand{\TrigEffPow}[1]{\ensuremath{\varepsilon^{#1}_{\tau}}\xspace}
\newcommand{\TrigDiEff}{\ensuremath{\varepsilon_{\DiTau}}\xspace}
\newcommand{\TrigDiEffSub}[1]{\ensuremath{\varepsilon_{\DiTau,~#1}}\xspace}
\newcommand{\TrigDiEffPow}[1]{\ensuremath{\varepsilon^{#1}_{\DiTau}}\xspace}

% Variable initialisation
\newcommand{\DefineTkTauFromCalo}[8]{
  \newcommand{\ldgTkPtMin}{#1}
  \newcommand{\tkCollectionType}{#2}
  \newcommand{\deltaRMcMatching}{#3}
  \newcommand{\sigConeDeltaR}{#4}
  \newcommand{\isoConeDeltaR}{#5}
  \newcommand{\ldgTkDeltaRCalo}{#6}
  \newcommand{\invMassMax}{#7}
  \newcommand{\deltaPOCAzMax}{#8}
}

%=============================================================================
%%% Particles: Quarks (qP) + Leptons (lP)
%=============================================================================
\newcommand{\Lepton}[1]{\ensuremath{\ell^{#1}}\xspace}
\newcommand{\Leptons}{\ensuremath{\ell = \mathrm{e}, \mu, \tau}\xspace}
\newcommand{\qQ}{\ensuremath{\mathrm{q}}} % q for quark
\newcommand{\qQdash}{\ensuremath{\mathrm{q}'}} % q for quark
\newcommand{\qAQ}{\ensuremath{\overline{\mathrm{q}}}} % anti-quark
\newcommand{\qAQdash}{\ensuremath{\overline{\mathrm{q}}'}} % anti-quark
\newcommand{\qU}{\ensuremath{\mathrm{u}}} % u for u quark
\newcommand{\qD}{\ensuremath{\mathrm{d}}} % d for d quark
\newcommand{\qC}{\ensuremath{\mathrm{c}}} % c for c quark
\newcommand{\qS}{\ensuremath{\mathrm{s}}} % s for s quark
\newcommand{\qT}{\ensuremath{\mathrm{t}}} % t for t quark
\newcommand{\qB}{\ensuremath{\mathrm{b}}} % b for b quark
\newcommand{\qAU}{\ensuremath{\overline{\mathrm{u}}}} % u for u anti-quark
\newcommand{\qAD}{\ensuremath{\overline{\mathrm{d}}}} % d for d anti-quark
\newcommand{\qAC}{\ensuremath{\overline{\mathrm{c}}}} % c for c anti-quark
\newcommand{\qAS}{\ensuremath{\overline{\mathrm{s}}}} % s for s anti-quark
\newcommand{\qAT}{\ensuremath{\overline{\mathrm{t}}}} % t for t anti-quark
\newcommand{\qAB}{\ensuremath{\overline{\mathrm{b}}}} % b for b anti-quark
\newcommand{\glu}{\ensuremath{\mathrm{g}}\xspace}
\newcommand{\Glu}{\glu}
\newcommand{\uQuark}{\ensuremath{\qU\text{-quark}}\xspace}
\newcommand{\dQuark}{\ensuremath{\qD\text{-quark}}\xspace}
\newcommand{\cQuark}{\ensuremath{\qC\text{-quark}}\xspace}
\newcommand{\sQuark}{\ensuremath{\qS\text{-quark}}\xspace}
\newcommand{\tQuark}{\ensuremath{\qT\text{-quark}}\xspace}
\newcommand{\bQuark}{\ensuremath{\qB\text{-quark}}\xspace}
\newcommand{\auQuark}{\ensuremath{\qAU\text{-quark}}\xspace}
\newcommand{\adQuark}{\ensuremath{\qAD\text{-quark}}\xspace}
\newcommand{\acQuark}{\ensuremath{\qAC\text{-quark}}\xspace}
\newcommand{\asQuark}{\ensuremath{\qAS\text{-quark}}\xspace}
\newcommand{\atQuark}{\ensuremath{\qAT\text{-quark}}\xspace}
\newcommand{\abQuark}{\ensuremath{\qAB\text{-quark}}\xspace}
\newcommand{\lL}[1]{\ensuremath{\mathrm{\ell}^{#1}}\xspace}   %lepton
\newcommand{\lE}[1]{\ensuremath{\mathrm{e}^{#1}}\xspace}      %electron
\newcommand{\lMu}[1]{\ensuremath{\mathrm{\mu}^{#1}}\xspace}   %muon
\newcommand{\lTau}[1]{\ensuremath{\mathrm{\tau}^{#1}}\xspace} %tau
\newcommand{\lNu}[1]{\ensuremath{\mathrm{\nu}_{#1}}\xspace}   %neutrino
\newcommand{\lANu}[1]{\ensuremath{\overline{\mathrm{\nu}}_{#1}}\xspace} %anti-neutrino
\newcommand{\lTauJet}{\ensuremath{\mathrm{\tau_{\textit{h}}}}\xspace}
\newcommand{\lTauJetMc}{\ensuremath{\mathrm{\tau_{\textit{h}, \mathrm{mc}}}}\xspace}
\newcommand{\jConeR}[1]{\ensuremath{\Delta\mathrm{R}_{#1}}\xspace}


%=============================================================================
%%% Particles: Mesons (\mP)
%=============================================================================
\newcommand{\mBMeson}[2]{\ensuremath{\mathrm{B}_{#1}^{#2}}\xspace}
\newcommand{\mJPsi}[1]{\ensuremath{\mathrm{J}/\hspace{-0.14em}\psi}} % J/Psi 
\newcommand{\mHadron}[1]{\ensuremath{\mathrm{h^{#1}}\xspace}} % h to represent pions and kaons
\newcommand{\mPion}[1]{\ensuremath{\mathrm{\pi^{#1}}\xspace}}
\newcommand{\mKaon}[2]{\ensuremath{\mathrm{K^{#1}_{#2}}\xspace}}
\newcommand{\mPionOrKaon}[1]{\ensuremath{\mathrm{h^{#1}}\xspace}}
\newcommand{\mRho}[1]{\ensuremath{\mathrm{\rho^{#1}}\xspace}}
\newcommand{\mAlpha}[1]{\ensuremath{\mathrm{\alpha_{1}^{#1}}\xspace}}

%=============================================================================
%%% Particles: Bosons (\bP)
%=============================================================================
\newcommand{\bPhoton}[1]{\ensuremath{\mathrm{\gamma}^{#1}}\xspace} % photon
\newcommand{\bW}[1]{\ensuremath{\mathrm{W}^{#1}}\xspace} % W
\newcommand{\bZ}[1]{\ensuremath{\mathrm{Z}^{#1}}\xspace} % Z
\newcommand{\bZPrime}[1]{\ensuremath{\mathrm{Z}^{#1\,\prime}}\xspace} % Z'
\newcommand{\bH}[1]{\ensuremath{\mathrm{H}^{#1}}\xspace}  % Higgs (H)
\newcommand{\bh}[1]{\ensuremath{\mathrm{h}^{#1}}\xspace}  % Higgs (h)
\newcommand{\bA}[1]{\ensuremath{\mathrm{A}^{#1}}\xspace}  % CP-odd Higgs (A)


%=============================================================================
%%% Particles: Pairs
%=============================================================================
\newcommand{\pTauTau}[2]{\ensuremath{\lTau{#1}\lTau{#2}}\xspace}

%=============================================================================
%%% Decay modes: Physics channels
%=============================================================================
\newcommand{\DiTau}{\ensuremath{\lTau{}\lTau{}}\xspace}
\newcommand{\WW}{\ensuremath{\bW{\pm}\bW{\mp}}\xspace}
\newcommand{\TTbar}{\ensuremath{\qT\qAT}\xspace}
\newcommand{\TTbarToBBWW}{\ensuremath{\qT\qAT \to \qB \bW{\pm} \qB \bW{\mp}}\xspace}
\newcommand{\HToTauTau}{\ensuremath{\bH{0} \to \lTau{+}\lTau{-}}\xspace}
\newcommand{\WToQQ}{\ensuremath{\bW{\pm} \to \qQ \qAQdash}\xspace}
\newcommand{\WToTauNu}{\ensuremath{\bW{\pm} \to \lTau{\pm} \lNu{\tau}}\xspace}
\newcommand{\HToTauNu}{\ensuremath{\bH{\pm} \to \lTau{\pm} \lNu{\tau}}\xspace}
%%% 1-prong tau decays
\newcommand{\TauToEle}{\ensuremath{\lTau{-}\to \lE{-} \lANu{e} \lNu{\tau} }\xspace}
\newcommand{\TauToMu}{\ensuremath{\lTau{-}\to \lMu{-} \lANu{\mu} \lNu{\tau} }\xspace}
\newcommand{\TauToOneProng}{\ensuremath{\lTau{-} \to \mHadron{-} \lNu{\tau} }\xspace}
\newcommand{\TauToOneProngRho}{\ensuremath{\lTau{-} \to \mRho{-} \lNu{\tau} \to \mHadron{-} \mPion{0} \lNu{\tau} }\xspace}
\newcommand{\TauToOneProngAlpha}{\ensuremath{\lTau{-} \to \mAlpha{-} \lNu{\tau} \to \mHadron{-} \mPion{0}  \mPion{0} \lNu{\tau} }\xspace}
\newcommand{\TauToOneProngGEThreePiZeros}{\ensuremath{\lTau{-} \to \mHadron{-} \lNu{\tau} + \geq 3\mPion{0}}\xspace}
%%% 3-prong tau decays
\newcommand{\TauToThreeProngAlpha}{\ensuremath{\lTau{-} \to \mAlpha{-} \lNu{\tau} \to \mHadron{-} \mHadron{+} \mHadron{-} \lNu{\tau} }\xspace}
\newcommand{\TauToThreeProngGEOnePiZero}{\ensuremath{\lTau{-} \to \mHadron{-} \mHadron{+} \mHadron{-} \lNu{\tau} + \geq 1\mPion{0}}\xspace}
%%% 5-prong tau decays
\newcommand{\TauToFiveProng}{\ensuremath{\lTau{-} \to \mHadron{-} \mHadron{+} \mHadron{-} \mHadron{+} \mHadron{-} \lNu{\tau} + \geq 0\mPion{0}}\xspace}
%%% Kaon tau decays 
\newcommand{\TauToKaonShort}{\ensuremath{\lTau{-} \to \mKaon{0}{S} + X}\xspace}
\newcommand{\TauToKaonLong}{\ensuremath{\lTau{-} \to \mKaon{0}{L} + X}\xspace}
%\newcommand{\TauPlusToPion}{\ensuremath{\tau^{+} \to \pi^{+} \bar{\nu}_{\tau} }\xspace}
%\newcommand{\TauMinusToPion}{\ensuremath{\tau^{-} \to \pi^{-} \nu_{\tau} }\xspace}
%\newcommand{\TaupmToPion}{\ensuremath{\tau^{\pm} \to \pi^{\pm} \nu_{\tau} }\xspace}
%\newcommand{\TauToThreeProngOnePiZero}{\ensuremath{\tau^{-} \to h^{-} h^{+} h^{-} \pi^{0} \nu_{\tau}}\xspace}


%============================================================================= 
%%% Symbols (s)
%============================================================================= 
\newcommand{\sRadiationLength}{\ensuremath{\mathrm{X}_{\mathrm{0}}}\xspace}
\newcommand{\sAbsorbtionLength}{\ensuremath{\lambda}\xspace}
\newcommand{\sAbsorbtionLengthFull}{\ensuremath{\lambda=\frac{1}{n \sigma_{\mathrm{inelastic}}}}\xspace}
\newcommand{\sMoliereRadius}{\ensuremath{\mathrm{R}_{\mathrm{m}}}\xspace}


%============================================================================= 
%%% PHYS 101 - Symbols (s)
%============================================================================= 
%\newcommand{\sUnitVector}[1]{\ensuremath{\vec{\hat{#1}}}\xspace} %no need for arrow. hat implies it (Ptochos)
\newcommand{\sUnitVector}[1]{\ensuremath{\hat{#1}}\xspace}
\newcommand{\sUnitVectorMag}[1]{\ensuremath{\abs{\sUnitVector{#1}}}\xspace}
\newcommand{\sDeltaDistance}[1]{\ensuremath{\Delta s_{#1}}\xspace}
\newcommand{\sDeltaDistancePow}[2]{\ensuremath{\Delta s_{#1}^{#2}}\xspace}
\newcommand{\sCoords}[2]{\ensuremath{\left(#1 , #2 \right) }\xspace}
\newcommand{\sCoordsTwoD}[2]{\ensuremath{\left(x_{#1} , y_{#2} \right) }\xspace}
\newcommand{\sCoordsRelativityTwoD}[2]{\ensuremath{\left(t_{#1} , x_{#2} \right) }\xspace}
\newcommand{\sLorentzFactor}{\ensuremath{\gamma}\xspace}
\newcommand{\sLorentzFactorPow}[1]{\ensuremath{\gamma^{#1}}\xspace}
\newcommand{\sRelVelocity}[1]{\ensuremath{\beta_{#1}}\xspace}
\newcommand{\sRelVelocityPow}[2]{\ensuremath{\beta_{#1}^{#2}}\xspace}
\newcommand{\sSpeedOfLight}{\ensuremath{c}\xspace}
\newcommand{\sGravitationalConstant}{\ensuremath{G}\xspace}
\newcommand{\sSpeedOfLightPow}[1]{\ensuremath{c^{#1}}\xspace}
\newcommand{\sDensity}[1]{\ensuremath{\rho_{#1}}\xspace}
\newcommand{\sMeanDensity}[1]{\ensuremath{\overline{\rho}_{#1}}\xspace}
\newcommand{\sVelocity}[1]{\ensuremath{v_{#1}}\xspace}
\newcommand{\sVelocityVec}[1]{\ensuremath{\vec{v}_{#1}}\xspace}
\newcommand{\sVelocityPow}[2]{\ensuremath{v_{#1}^{#2}}\xspace}
\newcommand{\sDeltaVelocity}[1]{\ensuremath{\Delta v_{#1}}\xspace}
\newcommand{\sDeltaVelocityVec}[1]{\ensuremath{\Delta \vec{v}_{#1}}\xspace}
\newcommand{\sDeltaVelocityPow}[2]{\ensuremath{\Delta v_{#1}^{#2}}\xspace}
\newcommand{\sMeanVelocity}[1]{\ensuremath{\overline{v}_{#1}}\xspace}
\newcommand{\sMeanVelocityVec}[1]{\ensuremath{\overline{\vec{v}}_{#1}}\xspace}
\newcommand{\sXYplane}{\ensuremath{x-y}\xspace}
\newcommand{\sPosition}[1]{\ensuremath{r_{\mathrm{#1}}}\xspace}
\newcommand{\sPositionX}[1]{\ensuremath{x_{\mathrm{#1}}}\xspace}
\newcommand{\sPositionY}[1]{\ensuremath{y_{\mathrm{#1}}}\xspace}
\newcommand{\sPositionZ}[1]{\ensuremath{z_{\mathrm{#1}}}\xspace}
\newcommand{\sMomentum}[1]{\ensuremath{p_{\mathrm{#1}}}\xspace}
\newcommand{\sDeltaMomentum}[1]{\ensuremath{\Delta p_{#1}}\xspace}
\newcommand{\sMomentumVec}[1]{\ensuremath{\vec{p}_{#1}}\xspace}
\newcommand{\sMomentumPow}[2]{\ensuremath{p_{#1}^{#2}}\xspace}
\newcommand{\sDeltaMomentumVec}[1]{\ensuremath{\Delta \vec{p}_{#1}}\xspace}
\newcommand{\sDeltaMomentumPow}[2]{\ensuremath{\Delta p_{#1}^{#2}}\xspace}
\newcommand{\sPositionPow}[2]{\ensuremath{r_{#1}^{#2}}\xspace}
\newcommand{\sPositionXPow}[2]{\ensuremath{x_{#1}^{#2}}\xspace}
\newcommand{\sPositionYPow}[2]{\ensuremath{y_{#1}^{#2}}\xspace}
\newcommand{\sPositionPowX}[2]{\ensuremath{x_{#1}^{#2}}\xspace}
\newcommand{\sPositionPowY}[2]{\ensuremath{y_{#1}^{#2}}\xspace}
\newcommand{\sDeltaPosition}[1]{\ensuremath{\Delta r_{#1}}\xspace}
\newcommand{\sDeltaPositionX}[1]{\ensuremath{\Delta x_{#1}}\xspace}
\newcommand{\sDeltaPositionPow}[2]{\ensuremath{\Delta x_{#1}^{#2}}\xspace}
\newcommand{\sDeltaPositionPowX}[2]{\ensuremath{\Delta x_{#1}^{#2}}\xspace}
\newcommand{\sDeltaPositionY}[1]{\ensuremath{\Delta y_{#1}}\xspace}
\newcommand{\sDeltaPositionPowY}[2]{\ensuremath{\Delta y_{#1}^{#2}}\xspace}
\newcommand{\sTime}[1]{\ensuremath{t_{#1}}\xspace}
\newcommand{\sMeanTime}[1]{\ensuremath{\overline{t}_{#1}}\xspace}
\newcommand{\sTimePow}[2]{\ensuremath{t_{#1}^{#2}}\xspace}
\newcommand{\sDeltaTime}[1]{\ensuremath{\Delta t_{#1}}\xspace}
\newcommand{\sDeltaTimePow}[2]{\ensuremath{\Delta t_{#1}^{#2}}\xspace}
\newcommand{\sAcceleration}[1]{\ensuremath{a_{#1}}\xspace}
\newcommand{\sAccelerationVec}[1]{\ensuremath{\vec{a}_{#1}}\xspace}
\newcommand{\sAccelerationPow}[2]{\ensuremath{a_{#1}^{#2}}\xspace}
\newcommand{\sGravity}[1]{\ensuremath{g_{#1}}\xspace}
\newcommand{\sGravityVec}[1]{\ensuremath{\vec{g}_{#1}}\xspace}
\newcommand{\sGravityValue}{\ensuremath{9.8 \sMeterSeconds{}{-2} }\xspace}
% Length
\newcommand{\sMeter}{\ensuremath{\xspace\text{m}}\xspace}
\newcommand{\sMeterPow}[2]{\ensuremath{\xspace\text{m}_{#1}^{#2}}\xspace}
\newcommand{\sKiloMeter}{\ensuremath{\xspace\text{k\hspace{-.05em}m}}\xspace}
\newcommand{\sMilliMeter}{\ensuremath{\xspace\text{m\hspace{-.05em}m}}\xspace}
\newcommand{\sCentiMeter}{\ensuremath{\xspace\text{c\hspace{-.05em}m}}\xspace}
\newcommand{\sCentiMeterPow}[1]{\ensuremath{\xspace\text{c\hspace{-.05em}m}^{#1}}\xspace}
\newcommand{\sMicroMeter}{\ensuremath{\xspace\text{$\mu$ \hspace{-0.35em}m}}\xspace}
\newcommand{\sNanoMeter}{\ensuremath{\xspace\text{n\hspace{-.05em}m}}\xspace}
\newcommand{\sAngstrom}{\mbox{\normalfont\AA}\xspace}
%Energy
\newcommand{\sMass}[1]{\ensuremath{\xspace m_{#1}}\xspace}
\newcommand{\sMassPow}[2]{\ensuremath{\xspace m_{#1}^{#2}}\xspace}
\newcommand{\sEnergy}[1]{\ensuremath{E_{#1}}\xspace}
\newcommand{\sEnergyPow}[2]{\ensuremath{E_{#1}^{#2}}\xspace}
\newcommand{\sDeltaEnergy}[1]{\ensuremath{\Delta E_{#1}}\xspace}
\newcommand{\sPower}[1]{\ensuremath{P_{#1}}\xspace}
\newcommand{\sPowerPow}[2]{\ensuremath{P_{#1}^{#2}}\xspace}
\newcommand{\sDeltaPower}[1]{\ensuremath{\Delta P_{#1}}\xspace}
\newcommand{\sWork}[1]{\ensuremath{W_{#1}}\xspace}
\newcommand{\sDeltaWork}[1]{\ensuremath{\Delta W_{#1}}\xspace}
% Time
\newcommand{\sWattMeters}[2]{\ensuremath{\xspace \text{W}^{#1} \hspace{-.05em}\text{m}^{#2}}\xspace}
\newcommand{\sIntensity}[1]{\ensuremath{I_{#1}}\xspace}
\newcommand{\sIntensityPow}[2]{\ensuremath{I_{#1}^{#2}}\xspace}
\newcommand{\sSecond}{\ensuremath{\xspace\text{s}}}
\newcommand{\sSeconds}{\sSecond}
\newcommand{\sHour}{\ensuremath{\xspace\text{h}}}
\newcommand{\sHours}{\sHour}
\newcommand{\sYear}{\ensuremath{\xspace\text{yr}}}
\newcommand{\sYears}{\ensuremath{\xspace\text{yrs}}}
\newcommand{\sSecondPow}[1]{\ensuremath{\xspace\text{s}^{#1}}\xspace}
\newcommand{\sSecondsPow}[1]{\sSecondPow{#1}}
\newcommand{\sMilliSecond}{\ensuremath{\xspace\text{m\hspace{-.05em}s}}\xspace}
\newcommand{\sMinute}{\ensuremath{\xspace\text{min}}\xspace}
\newcommand{\sMinuteCentimeter}[2]{\ensuremath{\xspace\text{min}^{#1}\text{cm}^{#2}}\xspace}
\newcommand{\sMicroSecond}{\ensuremath{\xspace\text{$\mu$ \hspace{-0.35em}s}}\xspace}
\newcommand{\sNanoSecond}{\ensuremath{\xspace\text{n \hspace{-0.35em}s}}\xspace}
% Mass
\newcommand{\sGram}{\ensuremath{\xspace\text{\hspace{-.05em}g}}\xspace}
\newcommand{\sMilliGram}{\ensuremath{\xspace\text{m\hspace{-.05em}g}}\xspace}
\newcommand{\sKiloGram}{\ensuremath{\xspace\text{k\hspace{-.05em}g}}\xspace}
\newcommand{\sKiloGramPow}[2]{\ensuremath{\xspace\text{k\hspace{-.05em}g}_{#1}^{#2}}\xspace}
% Frequency
\newcommand{\sHz}{\ensuremath{\xspace\text{\hspace{-.05em}Hz}}\xspace}
\newcommand{\sHertz}{\sHz}
\newcommand{\skHz}{\ensuremath{\xspace\text{k\hspace{-.05em}Hz}}\xspace}
\newcommand{\sMHz}{\ensuremath{\xspace\text{M\hspace{-.05em}Hz}}\xspace}
% Force
\newcommand{\sWeight}[1]{\ensuremath{W_{#1}}\xspace}
\newcommand{\sNormal}[1]{\ensuremath{N_{#1}}\xspace}
\newcommand{\sForce}[1]{\ensuremath{F_{#1}}\xspace}
\newcommand{\sForceVec}[1]{\ensuremath{\vec{F}_{#1}}\xspace}
\newcommand{\sForceVecMag}[1]{\ensuremath{\abs{\vec{F}_{#1}}}\xspace}
\newcommand{\sMeanForce}[1]{\ensuremath{\overline{F}_{#1}}\xspace}
\newcommand{\sTension}[1]{\ensuremath{T_{#1}}\xspace}
\newcommand{\sTensionVec}[1]{\ensuremath{\vec{T}_{#1}}\xspace}
% Circular Motion
\newcommand{\sRadius}[1]{\ensuremath{r_{#1}}\xspace} %R
\newcommand{\sRadiusPow}[2]{\ensuremath{r_{#1}^{#2}}\xspace} 
\newcommand{\sDeltaRadius}[1]{\ensuremath{\Delta r_{#1}}\xspace} %R
\newcommand{\sDeltaRadiusPow}[2]{\ensuremath{\Delta r_{#1}^{#2}}\xspace} %
\newcommand{\sOmega}[1]{\ensuremath{\omega_{#1}}\xspace}
\newcommand{\sOmegaPow}[2]{\ensuremath{\omega_{#1}^{#2}}\xspace}
\newcommand{\sTheta}[1]{\ensuremath{\theta_{#1}}\xspace}
\newcommand{\sThetaPow}[2]{\ensuremath{\theta_{#1}^{#2}}\xspace}
\newcommand{\sDeltaTheta}[1]{\ensuremath{\Delta \theta_{#1}}\xspace}
\newcommand{\sPeriod}[1]{\ensuremath{T_{#1}}\xspace}
\newcommand{\sPeriodPow}[2]{\ensuremath{T_{#1}^{#2}}\xspace}
\newcommand{\sFrequency}[1]{\ensuremath{f_{#1}}\xspace}
\newcommand{\sFrequencyPow}[2]{\ensuremath{f_{#1}^{#2}}\xspace}
\newcommand{\sWavelength}[1]{\ensuremath{\lambda_{#1}}\xspace}
\newcommand{\sWavelengthPow}[2]{\ensuremath{\lambda_{#1}^{#2}}\xspace}
\newcommand{\sWavenumber}[1]{\ensuremath{k_{#1}}\xspace}
\newcommand{\sWavenumberPow}[2]{\ensuremath{k_{#1}^{#2}}\xspace}
\newcommand{\sArc}[1]{\ensuremath{s_{#1}}\xspace}
\newcommand{\sDeltaArc}[1]{\ensuremath{\Delta s_{#1}}\xspace}
% Position  vectors
\newcommand{\sPosVec}[1]{\ensuremath{\vec{r}_{#1}}\xspace}
\newcommand{\sDeltaPosVec}[1]{\ensuremath{\Delta \vec{r}_{#1}}\xspace}
% Electricity / Magnetism
\newcommand{\sElectronCharge}{\lE{}\xspace}
\newcommand{\sVolt}[1]{\ensuremath{V_{#1}}\xspace}
\newcommand{\sEField}[1]{\ensuremath{E_{#1}}\xspace}
\newcommand{\sEFieldVec}[1]{\ensuremath{\vec{E}_{#1}}\xspace}
\newcommand{\sEFieldVecMag}[1]{\ensuremath{\abs{\vec{E}_{#1}}}\xspace}
\newcommand{\sCharge}[1]{\ensuremath{q_{#1}}\xspace}
\newcommand{\sChargePow}[2]{\ensuremath{q_{#1}^{#2}}\xspace}
\newcommand{\sChargeAlt}[1]{\ensuremath{Q_{#1}}\xspace}
\newcommand{\sChargeAltPow}[2]{\ensuremath{Q_{#1}^{#2}}\xspace}
\newcommand{\sBField}[1]{\ensuremath{B_{#1}}\xspace}
\newcommand{\sBFieldPow}[2]{\ensuremath{B_{#1}^{#2}}\xspace}
\newcommand{\sBFieldVec}[1]{\ensuremath{\vec{B}_{#1}}\xspace}
\newcommand{\sBFieldVecMag}[1]{\ensuremath{\abs{\vec{B}_{#1}}}\xspace}
% Relativity
\newcommand{\sBeta}[1]{\ensuremath{\beta_{#1} }\xspace}
\newcommand{\sBetaPow}[2]{\ensuremath{\beta_{#1}^{#2}}\xspace}
%Tracker
\newcommand{\sRInv}{\ensuremath{\rho^{-1}}\xspace}
\newcommand{\sLumiUnits}{\ensuremath{\text{cm}^\text{$-$2}\,\text{s}^\text{$-$1}}\xspace}
\newcommand{\sInvFb}{\mbox{\ensuremath{\,\text{fb}^\text{$-$1}}}\xspace}
           
%============================================================================= 
%%% Units
%============================================================================= 
\newcommand{\eV}{\ensuremath{\xspace\text{eV}}\xspace}
\newcommand{\MeV}{\ensuremath{\xspace\text{MeV}}\xspace}
\newcommand{\GeV}{\ensuremath{\xspace\text{GeV}}\xspace}
\newcommand{\GeVPow}[1]{\ensuremath{\xspace\text{GeV}^{\text{#1}}}\xspace}
\newcommand{\TeVc}[1]{\ensuremath{\xspace\text{TeV}c^\text{#1}}\xspace}
\newcommand{\GeVc}[1]{\ensuremath{\xspace\text{GeV}c^\text{#1}}\xspace}
\newcommand{\MeVc}[1]{\ensuremath{\xspace\text{MeV}c^\text{#1}}\xspace}

\newcommand{\sWatt}{\ensuremath{\xspace\text{W}}\xspace}
\newcommand{\sAmpere}{\ensuremath{\xspace\text{A}}\xspace}
\newcommand{\sAmperePow}[1]{\ensuremath{\xspace\text{A}^{#1}}\xspace}
\newcommand{\sCoulomb}{\ensuremath{\xspace\text{C}}\xspace}
\newcommand{\sCoulombPow}[2]{\ensuremath{\xspace\text{C}_{#1}^{#2}}\xspace}
\newcommand{\sCoulombConstant}{\ensuremath{\xspace\text{k}_{e}}\xspace}
\newcommand{\sPlanckConstant}{\ensuremath{h}\xspace}
\newcommand{\sHBar}{\ensuremath{\hbar}\xspace}
\newcommand{\sSpringConstant}{\ensuremath{\xspace\text{k}_{\text{ελ}}}\xspace}
\newcommand{\sNewton}{\ensuremath{\xspace\text{N}}\xspace}
\newcommand{\sJoule}{\ensuremath{\xspace\text{J}}\xspace}
\newcommand{\sDegrees}{\ensuremath{^{\circ}}\xspace}
\newcommand{\sRads}{\ensuremath{\text{rads}}\xspace}
\newcommand{\sMilliRads}{\ensuremath{\xspace\text{m\hspace{-.05em}\text{rads}}}\xspace}
\newcommand{\sAU}{\ensuremath{\en\text{au} }\xspace}
\newcommand{\sEV}{\ensuremath{\xspace\text{\hspace{-.05em}eV}}\xspace}
\newcommand{\sEVC}[2]{\ensuremath{\xspace\text{\hspace{-.05em}eV}^{#1} \text{c}^{#2} }\xspace}
\newcommand{\sT}{\ensuremath{\xspace\text{T}}\xspace}
\newcommand{\sTesla}{\sT}
% Combinations
\newcommand{\sJouleSeconds}[2]{\ensuremath{\xspace\text{J}^{#1}\hspace{-.05em}\text{s}^{#2}}\xspace}
\newcommand{\sMeterSeconds}[2]{\ensuremath{\xspace\text{m}^{#1}\hspace{-.05em}\text{s}^{#2}}\xspace}
\newcommand{\sNewtonMeter}[2]{\ensuremath{\xspace\text{N}^{#1}\hspace{-.05em}\text{m}^{#2}}\xspace}
\newcommand{\sNewtonKilograms}[2]{\ensuremath{\xspace\text{N}^{#1}\hspace{-.05em}\text{kg}^{#2}}\xspace}
\newcommand{\sGramMeter}[2]{\ensuremath{\xspace\text{g}^{#1}\hspace{-.05em}\text{m}^{#2}}\xspace}
\newcommand{\sGramCentiMeter}[2]{\ensuremath{\xspace\text{g}^{#1}\hspace{-.05em}\text{cm}^{#2}}\xspace}
\newcommand{\sKilogramMeterSeconds}[3]{\ensuremath{\xspace\text{kg}^{#1}\hspace{-.05em}\text{m}^{#2}\hspace{-.05em}\text{s}^{#3}}\xspace}
\newcommand{\sKilogramMeters}[2]{\ensuremath{\xspace\text{kg}^{#1}\hstopace{-.05em}\text{m}^{#2}}\xspace}
\newcommand{\sKiloMeterHours}[2]{\ensuremath{\xspace\text{km}^{#1}\hspace{-.05em}\text{h}^{#2}}\xspace}
\newcommand{\sKiloMeterSeconds}[2]{\ensuremath{\xspace\text{km}^{#1}\hspace{-.05em}\text{s}^{#2}}\xspace}
\newcommand{\sRadSeconds}[2]{\ensuremath{\xspace\text{rad}^{#1} \text{s}^{#2}}\xspace}
\newcommand{\sNewtonMeterKilograms}[3]{\ensuremath{\xspace\text{N}^{#1}\hspace{-.05em}\text{m}^{#2}\hspace{-.05em}\text{kg}^{#3}}\xspace}
\newcommand{\sNewtonMeterKiloGram}[3]{\sNewtonMeterKilograms{#1}{#2}{#3}}
\newcommand{\sNewtonMeterCoulombs}[3]{\ensuremath{\xspace\text{N}^{#1}\hspace{-.05em}\text{m}^{#2}\hspace{-.05em}\text{C}^{#3}}\xspace}
\newcommand{\sAmpereSeconds}[2]{\ensuremath{\xspace\text{A}^{#1} \text{s}^{#2}}\xspace}
\newcommand{\sVoltMeters}[2]{\ensuremath{\xspace\text{V}^{#1} \text{m}^{#2}}\xspace}


%============================================================================= 
%%% PHYS 101 - Elements (e)
%============================================================================= 
\newcommand{\eKrypton}{\ensuremath{\tensor[_{86}]{\text{Kr}}{}}\xspace}
\newcommand{\eCaesium}{\ensuremath{\tensor[_{55}]{\text{Cs}}{}}\xspace}
\newcommand{\eLead}{\ensuremath{\tensor[_{82}]{\text{Pb}}{}}\xspace}
\newcommand{\eSilicon}{\ensuremath{\tensor[_{14}]{\text{Si}}{}}\xspace}
\newcommand{\eSiliconAlt}{Si\xspace}

%============================================================================= 
%%% Tables
%============================================================================= 
\newcommand{\HLine}{\hline}

%============================================================================= 
%%% Constants
%============================================================================= 
\newcommand{\ceV}{\ensuremath{1.6 \times 10^{-19}\sJoule}\xspace}
\newcommand{\cGravity}{\ensuremath{9.8\sMeterSeconds{}{-2}}\xspace}
\newcommand{\cGravitationalConstant}{\ensuremath{6.67\times10^{-11}\sNewtonMeterKiloGram{}{2}{-2}}\xspace}
\newcommand{\cMassSun}{\ensuremath{2\times10^{30}\sKiloGram}\xspace}
\newcommand{\cMassEarth}{\ensuremath{6\times10^{24}\sKiloGram}\xspace}
\newcommand{\cMassProton}{\ensuremath{1.67\times10^{-27}\sKiloGram}\xspace}
\newcommand{\cMassElectron}{\ensuremath{9.11\times10^{-31}\sKiloGram}\xspace}
\newcommand{\cChargeElectron}{\ensuremath{1.6\times10^{-19}\sCoulomb}\xspace}
\newcommand{\cCoulombConstant}{\ensuremath{8.99\times10^{9} \sNewtonMeterCoulombs{}{2}{-2}}\xspace}
\newcommand{\cBohrRadius}{\ensuremath{0.53\times10^{-10}\sMeter}\xspace}
\newcommand{\cSpeedOfLight}{\ensuremath{3\times10^{8}\sMeterSeconds{}{-1}}\xspace}
\newcommand{\cRadiusSun}{\ensuremath{7\times10^{8}\sMeter}\xspace}
\newcommand{\cRadiusEarth}{\ensuremath{6.4\times10^{6}\sMeter}\xspace}
\newcommand{\cDistanceEarthSun}{\ensuremath{1.5\times10^{11} \sMeter}\xspace} %1AU
\newcommand{\cAU}{\cDistanceEarthSun\xspace}
\newcommand{\cPlanckConstant}{\ensuremath{6.64\times10^{-34}\sJouleSeconds{}{}}\xspace}
\newcommand{\cElectronCharge}{\ensuremath{1.602\times10^{-19}\sCoulomb}\xspace}
\newcommand{\cProtonCharge}{\cElectronCharge\xspace}
\newcommand{\cRadiusHydrogenAtom}{\ensuremath{\sRadius{} = 5.29\times10^{-11}\sMeter}\xspace}

%============================================================================= 
%%% Calculus
%============================================================================= 
\newcommand{\Derivative}[1]{\ensuremath{d#1}\xspace}
\newcommand{\DerivativeFrac}[2]{\ensuremath{\frac{d#1}{d#2}}\xspace}
\newcommand{\pDerivative}[1]{\ensuremath{\partial#1}\xspace}
\newcommand{\pDerivativeFrac}[2]{\ensuremath{\frac{\partial#1}{\partial#2}}\xspace}
\newcommand{\DerivativeSecondFrac}[2]{\ensuremath{\frac{d^{2}#1}{d#2^{2}}}\xspace}
\newcommand{\TimeFunc}[1]{\ensuremath{#1\left(t\right)}\xspace}
\newcommand{\TimeFuncSub}[2]{\ensuremath{#1_{#2}\left(t\right)}\xspace}
\newcommand{\TimeDerivative}[1]{\ensuremath{\dot{#1}\left(t\right)}\xspace}          %Newton's dot notation
\newcommand{\TimeDerivativeVec}[1]{\ensuremath{\vec{\dot{#1}}\left(t\right)}\xspace} %Newton's dot notation
\newcommand{\TimeDerivativeSecond}[1]{\ensuremath{\ddot{#1}}\left(t\right)\xspace}          %Newton's dot notation
\newcommand{\TimeDerivativeSecondVec}[1]{\ensuremath{\vec{\ddot{#1}}\left(t\right)}\xspace} %Newton's dot notation
\newcommand{\Func}[4]{\ensuremath{#1_{#3}^{#4}\left(#2\right)}\xspace}


%============================================================================= 
%%% Hyperlinks
%============================================================================= 
\newcommand{\hMCSamplesTP}[1]{\href{https://twiki.cern.ch/twiki/bin/viewauth/CMS/PdmVProductionUpgrade2014$\#$2\_Requests\_for\_the\_TTI\_Track\_Tri}{#1}}
\newcommand{\hMCSamplesFallTwentyThirteen}[1]{\href{https://cms-pdmv.cern.ch/mcm/requests?member_of_campaign=UpgFall13d&page=0}{#1}}

%============================================================================= 
%%% Matrices
%============================================================================= 
\newcommand{\IdentityMatrix}[1]{\ensuremath{\mathbb{I}_{#1}}\xspace}
\newcommand{\TransposeMatrix}[1]{\ensuremath{{#1}^{\mathrm{T}}}\xspace}
\newcommand{\AdjointMatrix}[1]{\ensuremath{adj\left(#1\right)}\xspace}
\newcommand{\DeterminantMatrix}[1]{\ensuremath{det\left(#1\right)}\xspace}
\newcommand{\InverseMatrix}[1]{\ensuremath{{#1}^{-1}}\xspace}

%=============================================================================
% Continuous and discrete random variables
%=============================================================================
\newcommand{\Expectation}[1]{\ensuremath{\text{E}\, [#1] }\xspace }
\newcommand{\Exp}[1]{\ensuremath{\overline{#1} }\xspace }
\newcommand{\mean}[1]{\ensuremath{\mu_{#1} }\xspace }
\newcommand{\Variance}[1]{\ensuremath{\text{Var}\,[#1] }\xspace}
\newcommand{\Var}[1]{\ensuremath{\sigma_{#1}^{2} }\xspace}
\newcommand{\StandDiv}[1]{\ensuremath{\text{SD\,} [#1] }\xspace}
\newcommand{\SD}[2]{\ensuremath{\sigma_{#1}^{#2} }\xspace}

\newcommand{\Covariance}[1]{\ensuremath{\text{Cov}\,[#1] }\xspace}
\newcommand{\Cov}[1]{\ensuremath{\sigma \left(#1\right)}\xspace}

\newcommand{\Correlation}[1]{\ensuremath{\rho\left(#1\right) }\xspace}
%\newcommand{\Corr}[1]{\ensuremath{\sigma \left(#1\right)}\xspace}

%=============================================================================
% Discrete random variables
%=============================================================================
% Binomial 
\newcommand{\kBnp}{\ensuremath{B_{k}\left(n, p\right)}\xspace}
\newcommand{\kBnpCustom}[2]{\ensuremath{B_{#1}\left(#2, p\right)}\xspace}
\newcommand{\kBnpFull}{\ensuremath{\nCk p^k \left(1-p\right)^{n-k}\xspace}}
\newcommand{\kBnpCustomFull}[2]{\ensuremath{\nCkCustom{#1}{#2} p^{#2} \left(1-p\right)^{#1-#2}\xspace}}
% Poisson
\newcommand{\kPlambda}{\ensuremath{P_{k}\left(\lambda\right)}\xspace}
\newcommand{\kPlambdaFull}{\ensuremath{\frac{\lambda^k \e{-\lambda}}{\Factorial{k}}}\xspace}
