%\newcommand{\en}{\selectlanguage{english}}
%\newcommand{\gr}{\selectlanguage{greek}}

\newcommand\PutLogo[3]{%
  \AtPageLowerLeft{%
    \put(\LenToUnit{#1\paperwidth},\LenToUnit{#2\paperheight}){#3}}%
}

\newcommand{\MakePosterLogos}[4]{ %%% do NOT remove '%' at line end`
  \AddToShipoutPictureFG{
    \PutLogo{0.010}{0.924}{\includegraphics[height=0.10\paperwidth,keepaspectratio]{./figures/logos/#1}}%
    \PutLogo{0.115}{0.924}{\includegraphics[height=0.10\paperwidth,keepaspectratio]{./figures/logos/#2}}%
    \PutLogo{0.785}{0.924}{\includegraphics[height=0.10\paperwidth,keepaspectratio]{./figures/logos/#3}}%
    \PutLogo{0.890}{0.924}{\includegraphics[height=0.10\paperwidth,keepaspectratio]{./figures/logos/#4}}%
%    \PutLogo{0.1}{0.925}{\includegraphics[width=0.10\paperwidth,keepaspectratio]{./figures/logos/#1}}%
%    \PutLogo{0.0}{0.925}{\includegraphics[width=0.10\paperwidth,keepaspectratio]{./figures/logos/#2}}%
%    \PutLogo{0.8}{0.925}{\includegraphics[width=0.10\paperwidth,keepaspectratio]{./figures/logos/#3}}%
%    \PutLogo{0.9}{0.925}{\includegraphics[width=0.10\paperwidth,keepaspectratio]{./figures/logos/#4}}%
  }
}

\newcommand{\MakePosterColumns}[3]{
\newlength{\ColumnWidth}
\setlength{\ColumnWidth}{0.315\paperwidth}
\begin{columns}[t]
    \begin{columns}[t,totalwidth=1.0\paperwidth] % split up that three-column-wide column
      \begin{column}{\ColumnWidth} \input{./tex/#1.tex} \end{column}
      \begin{column}{\ColumnWidth} \input{./tex/#2.tex} \end{column}
      \begin{column}{\ColumnWidth} \input{./tex/#3.tex} \end{column}    
    \end{columns}
  \end{columns}
}

% =======================================================================
% Colour definitions. See: http://en.wikipedia.org/wiki/Web_colors
% =======================================================================
\definecolor{kOldPaper}   {RGB}{231, 229, 220}
\definecolor{kOldPaperV1} {RGB}{250, 252, 243}
\definecolor{kOldPaperAlt}{RGB}{254, 241, 226}
\definecolor{kBlack}      {RGB}{  0,   0,   0}
\definecolor{kWhite}      {RGB}{255, 255, 255}
\definecolor{kPink}       {RGB}{227,  74, 147}
\definecolor{kNavyBlue}   {RGB}{ 28, 130, 185}
\definecolor{kDarkBlue}   {RGB}{  0,   0, 214}
\definecolor{kLightBlue}  {RGB}{  0,  61, 245}
\definecolor{kvLightBlue} {RGB}{  0, 184, 245}
\definecolor{kMyBlue}     {RGB}{ 51, 102, 255}
\definecolor{kHLTausBlue} {RGB}{ 11,  36, 251}
\definecolor{kBlue}   {RGB}{ 51, 102, 255}
\definecolor{kGreen}  {RGB}{ 98, 158,  31}
\definecolor{kLightGreen}{RGB}{223, 254,  191}
\definecolor{kRed}    {RGB}{220,   0,   0}
\definecolor{kOrange} {RGB}{230, 120,  20}
\definecolor{kYellow} {RGB}{255, 221,   0}
\definecolor{kBlue}   {RGB}{ 10,  50, 150}
\definecolor{kBrown}  {RGB}{120,  89,  30}
\definecolor{kPink}   {RGB}{255,   0, 128}

% =======================================================================
% Customise tcolorboxes (package "tcolorbox")
% =======================================================================
\tcbset{
    noparskip,
    coltitle  = green,
    colframe  = pink,
    colback   = orange,
    coltext   = yellow,
    fonttitle =\normalsize, 
    posterStyle/.style = {coltitle=red, colframe=brown, colback=brown, coltext=red, fonttitle = \Large},
    boxsep       = +0.20cm,   % Title Box Height
    top          = +0.50cm,   % Top Margin (from Title box line)
    bottom       = +0.00cm,   % Bottom Margin 
    left         = +0.10cm,   % Left Margin
    right        = +0.10cm,   % Right Margin
    toptitle     = +0.10cm,   % ?
    bottomtitle  = +0.10cm,   % ?
    arc          = +0.8cm,
    nobeforeafter,
    %center title,
}
    
% new tcolorbox environment
\newtcolorbox{headline}[2][]{
  coltext          = black,
  colframe         = black, %gray!50
  colback          = \MyBlockFillColorRight,
  colbacktitle     = \MyBlockTitleBoxColor,
  coltitle         = black,
  title            = #2,
  fonttitle        = \bfseries\Huge,
  boxrule          = 0pt, % border line width
  % opacityframe     = 0,   % remove box border line
  titlerule       = 1pt,
  % colbacktitle    = red!50!yellow,
  % titlerule style = black,
  titlerule style = {black, arrows = {Hooks [arc=270]-Hooks [arc=270]}},
  opacityback     = 1.0, % 1.0 means totally transparent, 0.0 means totally opaque
  top             =-0.0cm,
  bottom          = 0.0cm,
  left            = 0.05cm,
  right           = 0.05cm,
  arc             = 0.0cm, % 0.0cm for non-rounded corners!
  sharp corners,
  frame hidden,
  parbox=false,
  #1,
}

% new tcolorbox environment
\newtcolorbox{SubArticle}[2][]{
  coltext      = black,
  colframe     = black, %gray!50
  colback      = \MyBlockFillColorRight,
  colbacktitle = \MyBlockTitleBoxColor,
  coltitle     = black,
  title        = #2,
  fonttitle    = \bfseries\large,
  boxrule      = 0pt, % border line width
  % opacityframe = 0,   % remove box border line
  titlerule    = 1pt,
  % colbacktitle    = red!50!yellow,
  % titlerule style = black,
  %titlerule style = {black, arrows = {Hooks [arc=270]-Hooks [arc=270]}},
  opacityback     = 1.0, % 1.0 means totally transparent, 0.0 means totally opaque
  top             = +0.00cm,
  bottom          = +0.00cm,
  left            = +0.05cm,
  right           = +0.05cm,
  arc             = +0.00cm, % 0.0cm for non-rounded corners!
  sharp corners,
  frame hidden,
  parbox=false,
  #1,
}

% new tcolorbox environment
\newtcolorbox{MyArticle}[2][]{
  parbox       = false,
  coltext      = black,
  colframe     = black, %gray!50, %\MyBlockFrameColorLeft,
  colback      = \MyBlockFillColorRight, %white, %\MyBlockFillColorLeft,
  colbacktitle = \MyBlockTitleBoxColor,
  coltitle     = black,
  title        = {\Large{\textbf{#2}}},
  fonttitle    = \bfseries,
  boxrule      = 0pt, %frame line width
  % borderline   = {1mm}{0mm}{solid}, % dashed  {dot width}{margin width}
  titlerule    = 0pt, %tiitle line width
  opacityback  = 1.0, % 1.0 means totally transparent, 0.0 means totally opaque
  top          = -0.20cm,
  bottom       = +0.00cm,
  left         = +0.05cm,
  right        = +0.05cm,
  arc          = +0.00cm, % 0.0cm for non-rounded corners!
  sharp corners,
  #1,
}

\newtcolorbox{MyAd}[1][]{
  parbox       = false,
  coltext      = black,
  colframe     = black, %gray!50, %\MyBlockFrameColorLeft,
  colback      = \MyBlockFillColorRight, %white, %\MyBlockFillColorLeft,
  boxrule      = 0pt, %frame line width
  opacityback  = 1.0, % 1.0 means totally transparent, 0.0 means totally opaque
  top          = -0.00cm,
  bottom       = +0.00cm,
  left         = +0.00cm,
  right        = +0.00cm,
  arc          = +0.00cm, % 0.0cm for non-rounded corners!
  sharp corners,
  #1,
}

% new tcolorbox environment
\newtcolorbox{MyArticleDotted}[2][]{
  coltext      = black,
  colframe     = black, %gray!50, %\MyBlockFrameColorLeft,
  colback      = \MyBlockFillColorRight, %white, %\MyBlockFillColorLeft,
  colbacktitle = \MyBlockTitleBoxColor,
  coltitle     = black,
  title        = {\Large{\textbf{#2}}},
  fonttitle    = \bfseries,
  boxrule      = 0pt, %frame line width
  borderline={1mm}{0mm}{dotted}, % dashed  {dot width}{margin width}
  titlerule    = 0pt, %tiitle line width
  %tikz={rotate=#3}, % manipulate the tcolorbox as a whole (in degrees)
  top=-0.0cm, bottom=+0.0cm, left=+0.05cm, right=+0.05cm,
  %enlarge top by   = +1.0cm,  %  equivalent to mdframed 'skipabove'
  %enlarge bottom by= +0.0cm,  %  equivalent to mdframed 'skipbelow'
  %enlarge left by  = +1.5cm,  
  %enlarge right by = +0.0cm, 
  opacityback=1.0, % 1.0 means totally transparent, 0.0 means totally opaque
  arc=0.0cm,        % 0.0cm for non-rounded corners!
  sharp corners,
  #1,
}

% new tcolorbox environment
\newtcolorbox{MyArticleDashed}[2][]{
  coltext      = black,
  colframe     = black, %gray!50, %\MyBlockFrameColorLeft,
  colback      = \MyBlockFillColorRight, %white, %\MyBlockFillColorLeft,
  colbacktitle = \MyBlockTitleBoxColor,
  coltitle     = black,
  title        = {\Large{\textbf{#2}}},
  fonttitle    = \bfseries,
  boxrule      = 0pt, %frame line width
  borderline={1mm}{0mm}{dashed}, % dashed  {dot width}{margin width}
  titlerule    = 0pt, %tiitle line width
  %tikz={rotate=#3}, % manipulate the tcolorbox as a whole (in degrees)
  top=-0.0cm, bottom=+0.0cm, left=+0.05cm, right=+0.05cm,
  %enlarge top by   = +1.0cm,  %  equivalent to mdframed 'skipabove'
  %enlarge bottom by= +0.0cm,  %  equivalent to mdframed 'skipbelow'
  %enlarge left by  = +1.5cm,  
  %enlarge right by = +0.0cm, 
  opacityback=1.0, % 1.0 means totally transparent, 0.0 means totally opaque
  arc=0.0cm,        % 0.0cm for non-rounded corners!
  sharp corners,
  #1,
}


\newtcolorbox{multimuons-1}[2][]{ 
  coltext      = black,
  colframe     = black, %\MyBlockFrameColorLeft,
  colback      = \MyBlockFillColorRight, %white, %\MyBlockFillColorLeft,
  colbacktitle = \MyBlockTitleBoxColor,
  coltitle     = black,
  title        = {\Large{\textbf{#2}}},
  fonttitle    = \bfseries,
  boxrule      = 0pt, %frame line width
  titlerule    = 0pt, %tiitle line width
  titlerule style = {white},
  %tikz={rotate=#3}, % manipulate the tcolorbox as a whole (in degrees)
  top=-0.0cm, bottom=+0.0cm, left=+0.05cm, right=+0.05cm,
  %enlarge top by   = +1.0cm,  %  equivalent to mdframed 'skipabove'
  %enlarge bottom by= +0.0cm,  %  equivalent to mdframed 'skipbelow'
  %enlarge left by  = +1.5cm,  
  %enlarge right by = +0.0cm, 
  opacityback=1.0, % 1.0 means totally transparent, 0.0 means totally opaque
  arc=0.0cm,        % 0.0cm for non-rounded corners!
  sharp corners,
  frame hidden,
  #1,
}

\newtcolorbox{multimuons-2}[2][]{ 
  coltext      = black,
  colframe     = black, %\MyBlockFrameColorLeft,
  colback      = \MyBlockFillColorRight, %white, %\MyBlockFillColorLeft,
  colbacktitle = \MyBlockTitleBoxColor,
  coltitle     = black,
  title        = {\Large{\textbf{#2}}},
  fonttitle    = \bfseries,
  boxrule      = 1pt, %frame line width
  titlerule    = 0pt, %tiitle line width
  titlerule style = {white},
  %tikz={rotate=#3}, % manipulate the tcolorbox as a whole (in degrees)
  top=-0.2cm, bottom=+0.0cm, left=+0.05cm, right=+0.05cm,
  %enlarge top by   = +1.0cm,  %  equivalent to mdframed 'skipabove'
  %enlarge bottom by= +0.0cm,  %  equivalent to mdframed 'skipbelow'
  %enlarge left by  = +1.5cm,  
  %enlarge right by = +0.0cm, 
  opacityback=1.0, % 1.0 means totally transparent, 0.0 means totally opaque
  arc=0.0cm,        % 0.0cm for non-rounded corners!
  sharp corners,
  frame hidden,
  #1,
}

% =======================================================================
% Block format/colour definitions
% =======================================================================
\newcommand{\CustomiseColours}[9]{
  \newcommand{\MyFontColor}{#1}
  \newcommand{\MyHeadingsColor}{#2}
  \newcommand{\MyBlockFillColorLeft}{#3}
  \newcommand{\MyBlockFrameColorLeft}{#3}
  \newcommand{\MyBlockFillColorCenter}{#4}
  \newcommand{\MyBlockFrameColorCenter}{#4}
  \newcommand{\MyBlockFillColorRight}{#5}
  \newcommand{\MyBlockFrameColorRight}{#5}
  \newcommand{\MyBlockTitleBoxColor}{#6}
  \newcommand{\MyBackgroundColour}{#7}
  \newcommand{\MyFooterFontColour}{#8}
  \newcommand{\MyFooterFillColour}{#9}
  \setbeamercolor{background canvas}{bg=\MyBackgroundColour}
  \setbeamertemplate{navigation symbols}{}
  \setbeamercolor{block title}{fg=\MyFontColor, bg=black}
  \setbeamercolor{block body} {fg=\MyBlockFillColor, bg=black} 
  \setbeamercolor{block alerted title}{fg=red, bg=pink}
  \setbeamercolor{block alerted body} {fg=red, bg=pink}
  \setbeamercolor{title in head/foot}{fg=\MyFooterFontColour, bg=\MyFooterFillColour}
}

% =======================================================================
% Footer
% =======================================================================
\setbeamercolor{title in head/foot}{fg=black, bg=\MyBackgroundColour}
\newcommand{\FooterHeight}{0.8cm}
\newcommand{\FooterBoxDepth}{0.6cm} % The depth of a box is the distance between the baseline and the bottom of the box;

% Footer
\newcommand{\SetFooterText}[1]{
  \setbeamertemplate{footline}{
    \leavevmode
    \hbox{\begin{beamercolorbox}[wd=1.0\paperwidth,ht=\FooterHeight, dp=\FooterBoxDepth, leftskip=0.01\paperwidth, rightskip=0.01\paperwidth]{title in head/foot}
        \usebeamerfont{title in head/foot}  \centering \normalsize{#1}
      \end{beamercolorbox}
    }
  }
}

% Footer (Alternative)
\newcommand{\CustomiseFooterAlt}[3]{
  \setbeamercolor{title in head/foot}{fg=#1,bg=#2}
  \setbeamertemplate{footline}{
    \leavevmode
    \hbox{\begin{beamercolorbox}[wd=1.0\paperwidth, ht=\FooterHeight, dp=\FooterBoxDepth]{title in head/foot}
        \usebeamerfont{title in head/foot}  \centering \normalsize{#3}
      \end{beamercolorbox}
    }
  }
}


% =======================================================================
% Figures
% =======================================================================
\newcommand{\oneSubFig}[3]{
  \includegraphics[width=#3\textwidth,keepaspectratio]{figures/#2}
  \label{fig:#1}
}

\newcommand{\oneFigPosterNoCaption}[3]{
  \begin{figure}[tbp]  
      \begin{columns}
        \begin{column}{0.99\textwidth}
          \centering
          \includegraphics[width=#3\textwidth,keepaspectratio]{figures/#2}
          \label{fig:#1}
        \end{column}
      \end{columns}
  \end{figure}
}


\newcommand{\oneFigPoster}[4]{
  \begin{figure}[tbp]
    \def\figurename{Εικόνα}
      \begin{columns}
        \begin{column}{0.99\textwidth}
          \centering
          \includegraphics[width=#3\textwidth,keepaspectratio]{figures/#2}
          \label{fig:#1}
        \end{column}
      \end{columns}
    \caption{#4}
  \end{figure}
}

\newcommand{\twoFigPoster}[6]{
  \begin{figure}[tbp]
    \def\figurename{Εικόνα}
      \begin{columns}
        \begin{column}{0.47\textwidth}
          \oneSubFig{#1_a}{#2}{#3}
        \end{column}
        \begin{column}{0.47\textwidth}
          \oneSubFig{#1_b}{#4}{#5}
        \end{column}
      \end{columns}
    \caption{#6}
  \end{figure}
}

\newcommand{\twoFigPosterNoCaption}[5]{
  \begin{figure}[tbp]
      \begin{columns}
        \begin{column}{0.47\textwidth}
          \oneSubFig{#1_a}{#2}{#3}
        \end{column}
        \begin{column}{0.47\textwidth}
          \oneSubFig{#1_b}{#4}{#5}
        \end{column}
      \end{columns}
  \end{figure}
}

\newcommand{\twoFigColumnsPoster}[6]{
  \begin{figure}[tbp]
    \begin{columns}
      \begin{column}{0.5\textwidth}
        \begin{tcolorbox}[posterStyle]
          \oneSubFig{#1_a}{#2}{#3}
    \end{tcolorbox}
      \end{column}
      \begin{column}{0.5\textwidth}
        \begin{tcolorbox}[posterStyle]
          \oneSubFig{#1_b}{#4}{#5}
    \end{tcolorbox}       
      \end{column}
  \end{columns}
    \caption{#6}
  \end{figure}
}

% =======================================================================
% Enumerate Customisations (TikZ)
% =======================================================================
\newcommand*\circled[2]{\tikz[baseline=(char.base)]{  \node[circle, ball color=#2, inner sep=0.1cm] (char) {\textcolor{kWhite}#1};}}
\newcommand*\rounded[2]{\tikz[baseline=(char.base)]{ \node[draw=none,ball color=#2, shade, rounded corners=3.5pt, inner sep=0.1cm] (char) {\textcolor{kWhite}{#1}};}}
