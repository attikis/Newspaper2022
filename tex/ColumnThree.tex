%%%%%%%%%%%%%%%%%%%%%%%%%%%%%%%%%%%%%%%%%%%%%%%%%%%%%%%%%%%%%%%%%%%%%%
%% LaTeX (Beamer) Document for Presentations 
%% File ..............: ColumnThree.tex
%% Author ............: Alexandros X. Attikis
%% Institute .........: University of Cyprus (UCY)
%% e-Mail ............: attikis@cern.ch
%% Comments ..........: 
%%%%%%%%%%%%%%%%%%%%%%%%%%%%%%%%%%%%%%%%%%%%%%%%%%%%%%%%%%%%%%%%%%%%%%

%%%%%%%%%%%%%%%%%%%%%%%%%%%%%%%%%%%%%%%%%%%%%%%%%%%%%%%%%%%%%%%%%%%%%%
%%% Block 7
%%%%%%%%%%%%%%%%%%%%%%%%%%%%%%%%%%%%%%%%%%%%%%%%%%%%%%%%%%%%%%%%%%%%%%
\begin{MyColumnRight}[detach title,before upper={\tcbtitle\quad}]{Πως φτάνουν τα μιόνια στο έδαφος$;$}{34.0cm}

  Έστω ότι ένα μιόνιο παράχθηκε σε ύψος \en10 \sKiloMeter \gr από την
  επιφάνεια της Γης με τη διάσπαση $\mPion{-} \to \lMu{-} +
  \lANu{\lMu{}}$ και κινείται προς τη Γη με
  ταχύτητα $\sVelocity{} =0.9999 \sSpeedOfLight$. Θα προλάβει το μιόνιο
  να κτυπήσει το έδαφος πριν διασπαστεί$;$\\ 

Η απόσταση που διανύει το μιόνιο είναι:
%%%%%%%%%%
\en
\begin{align*}
\sDeltaPositionX{} &= \sVelocity{} \sDeltaTime{} = \left(0.9999
\sSpeedOfLight\right) \cdot \left(2 \,\sMicroSecond\right) = 600 \,\sMeter
\end{align*}
\gr
%%%%%%%%%%
Εφόσον \en600 \sMeter $\ll$ 10 \sKiloMeter\gr το μιόνιο δεν θα φτάσει
ποτέ στο έδαφος! Και όμως, βλέπουμε μιόνια να φτάνουν στον θαλάμο νέφωσης! Πως γίνεται αυτό$;$ \\

Τη λύση μας την δίνει η Ειδική Θεωρία της Σχετικότητας του Άινσταιν που
βασίζεται σε δύο αξιώματα:\\
\circled{1}{\MyHeadingsColor} Η ταχύτητα του φωτός είναι η ίδια για όλους τους αδρανειακούς παρατηρητές, ανεξάρτητα από την
σχετική τους κίνηση.
\\
\circled{2}{\MyHeadingsColor} Οι νόμοι της φυσικής είναι αμετάβλητοι σε όλα τα αδρανειακά συστήματα αναφοράς.
\\

Οι συνέπειες των αξιωμάτων αυτών είναι:\\
\circled{1}{\MyHeadingsColor} Διαστολή Χρόνου: Ένας παρατηρητής βλέπει το ρολόι που φοράει ένας
κινούμενος παρατηρητής να αργοπορεί σε σχέση με το δικό του.% ρολόι.
\\
\circled{2}{\MyHeadingsColor} Συστολή Μήκους: Οι διαστάσεις των αντικειμένων συστέλλονται κατά
μήκος της διεύθυνσης κίνησης τους, σύμφωνα με ένα σχετικά ακίνητο παρατηρητή.
\\

Οποιαδήποτε μέτρηση μεγέθους κίνησης, όπως του χώρου και του χρόνου,
δεν έχει κανένα νόημα αν δεν ξέρουμε ως προς τι μετριέται. Τόσο ο χρόνος όσο και το μήκος είναι σχετικές ποσότητες!

\end{MyColumnRight}
%%%%%%%%%%%%%%%%%%%%%%%%%%%%%%%%%%%%%%%%%%%%%%%%%%%%%%%%%%%%%%%%%%%%%%


%%%%%%%%%%%%%%%%%%%%%%%%%%%%%%%%%%%%%%%%%%%%%%%%%%%%%%%%%%%%%%%%%%%%%%
%%% Block 8
%%%%%%%%%%%%%%%%%%%%%%%%%%%%%%%%%%%%%%%%%%%%%%%%%%%%%%%%%%%%%%%%%%%%%%
\begin{MyColumnRight}[detach title,before upper={\tcbtitle\quad}]{Σύστημα αναφοράς κοσμικού μιονίου}{34.0cm}
%{Προσέγγιση Συστολής Χώρου - Σύστημα Αναφοράς Κοσμικού Μιονίου}

Tο πάχος ατμόσφαιρας αναφέρεται στο πάχος όπως μετριέται στο σύστημα
ενός παρατηρητή στη Γη. Αντίθετα, ο χρόνος ζωής του μιονίου αναφέρεται
στο χρόνο ζωής όπως τον αντιλαμβάνεται ένας παρατηρητής που κινείται
με το μιόνιο. %Δηλαδή, ο χρόνος ζωής του μιονίου \emph{ορίζεται} στο σύστημα ηρεμίας του μιονίου. 
Σε αυτό το σύστημα αναφοράς, θεωρούμε ότι αντί να κινείται το μιόνιο, το έδαφος πλησιάζει το μιόνιο
με ταχύτητα $\sVelocity{} =0.9999 \sSpeedOfLight$. 
Έτσι, σε αυτό το χρονικό διάστημα το έδαφος διανύει μια απόσταση:
%%%%%%%%%%
\en
\begin{align*}
\sDeltaPositionX{} &= \sVelocity{} \sDeltaTime{} = \left(0.9999 \sSpeedOfLight\right) \cdot \left(2\,\sMicroSecond\right) = 600 \,\sMeter
\end{align*}
\gr
%%%%%%%%%%
πριν να διασπαστεί το μιόνιο. \\

Ποιό είναι όμως το πάχος της  ατμόσφαιρας που βλέπει το μιόνιο$;$ Λόγω
του φαινομένου της συστολής μήκους, το μήκος της ατμόσφαιρας όπως
μετριέται από παρατηρητή που βρίσκεται στο σύστημα αναφοράς του μιονίου
μικραίνει κατά ένα παράγοντα:
%%%%%%%%%%
\en
\begin{align*}
\sLorentzFactor = \frac{1}{\sqrt{1-(0.9999)^{2}}} = 70.7
\end{align*}
\gr
%%%%%%%%%%
Έτσι, το πάχος της ατμόσφαιρας που βλέπει το μιόνιο είναι 70.7 φορές
πιο μικρό από ότι μετράει ο παρατηρητής στη Γη. Άρα, σύμφωνα με το μιόνιο η
ατμόσφαιρα φαίνεται να έχει πάχος:
%%%%%%%%%%
\en
\begin{align*}
\sDeltaPositionX{} &= \frac{\sDeltaPositionX{0}}{\sLorentzFactor} = \frac{10\,\sKiloMeter}{70.7} = 141 \,\sMeter 
\end{align*}
\gr
%%%%%%%%%%  
όπου \sDeltaPositionX{0} το μήκος ηρεμίας της απόσταστη Γης-μιονίου,
όταν αυτό παράχθηκε. \\

Εφόσον \en600 \sMeter $>$ 141 \sMeter\gr, το έδαφος θα φτάσει το μιόνιο πριν αυτό διασπαστεί!

\end{MyColumnRight}
%%%%%%%%%%%%%%%%%%%%%%%%%%%%%%%%%%%%%%%%%%%%%%%%%%%%%%%%%%%%%%%%%%%%%%


%%%%%%%%%%%%%%%%%%%%%%%%%%%%%%%%%%%%%%%%%%%%%%%%%%%%%%%%%%%%%%%%%%%%%%
%%% Block 9
%%%%%%%%%%%%%%%%%%%%%%%%%%%%%%%%%%%%%%%%%%%%%%%%%%%%%%%%%%%%%%%%%%%%%%
\begin{MyColumnRight}[detach title,before upper={\tcbtitle\quad}]{Σύστημα αναφοράς παρατηρητή στη Γη}{34.0cm}
%{Προσέγγιση Διαστολής Χρόνου - Σύστημα Αναφοράς Γήινου Παρατηρητή}

Προσεγγίζουμε τα μιόνια σαν ρολόγια στο σύστημα αναφοράς του μιονίου.
Τη χρονική στιγμή \sTime{} = 0 το μιόνιο παράγεται και τη χρονική
στιγμή \sTime{} = \lTau{} (μέσος χρόνος ζωής του μιονίου) το μιόνιο διασπάται. 
Τα δύο γεγονότα  λαμβάνουν χώρα στο ίδιο σημείο, αφού για ένα
παρατηρητή που κινείται με το μιόνιο, αυτό παραμένει ακίνητο.\\

Ένας ακίνητος παρατηρητής στη Γη μετράει τη χρονική απόσταση αυτών των γεγονότων ως: 
%%%%%%%%%%
\en
\begin{align*}
\sDeltaTime{} &= \sLorentzFactor \sDeltaTime{0} = \sLorentzFactor \lTau{} \\
\sDeltaTime{} &= \left(70.7\right) \cdot  \left(2 \,\sMicroSecond\right) \\
\sDeltaTime{} &= 141.4 \,\sMicroSecond
\end{align*}
\gr
%%%%%%%%%%

Δηλαδή, ο χρόνος ζωής του μιονίου φαίνεται να είναι 70.7 φορές
μεγαλύτερος όταν μετρηθεί από ένα παρατηρητή στη Γη!\\

Έτσι, ο παρατηρητής στη Γη βλέπει το μιόνιο να διανύει μια απόσταση:
%%%%%%%%%%
\en
\begin{align*}
\sDeltaPositionX{} & = \sVelocity{} \sDeltaTime{} = \left(0.9999 \sSpeedOfLight\right) \cdot \left(141 \times 10^{-6} \,\sSecond\right) \\
\sDeltaPositionX{}  &= 42296 \,\sMeter \\
\sDeltaPositionX{}  &\cong 42 \,\sKiloMeter
\end{align*}
\gr
%%%%%%%%%%
πριν αυτό διασπαστεί. Εφόσον \en42 \sKiloMeter $>$ 10 \sKiloMeter\gr, το μιόνιο θα φτάσει το έδαφος πριν αυτό διασπαστεί!\\

Βρίσκουμε πως οι δύο παρατηρητές βγάζουν ακριβώς το ίδιο συμπέρασμα:
το μιόνιο θα φτάσει στην επιφάνεια της Γης πριν διασπαστεί!
Oι νόμοι της φυσικής λοιπόν είναι οι ίδιοι για όλους και ανεξάρτητοι από το σύστημα αναφοράς
που διαλέγουμε για να λύσουμε ένα πρόβλημα!

\end{MyColumnRight}
%%%%%%%%%%%%%%%%%%%%%%%%%%%%%%%%%%%%%%%%%%%%%%%%%%%%%%%%%%%%%%%%%%%%%%
