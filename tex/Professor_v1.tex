\begin{headline}[enhanced, tikz={rotate=0}]{Fotios Ptochos Promoted to Professor!}
  \begin{multicols}{2}
 Congratulations to Fotios Ptochos for his promotion to the rank of
 Full Professor, effective November 2022. This promotion recognizes
 Prof. Ptochos' achievements in scholarship, teaching in physics and research in
 high-energy physics (HEP), and his overall service to the CDF and CMS
 Collaborations. He is a Harvard University PhD in physics
 graduate (1998) and has been active in HEP-research, both in detector
development and physics analyses since 1987. In particular, from 1987
to 1988 he worked in the development of a technique to monitor the
purity of Liquid Argon (LAr) for the first ever prototype of the
ICARUS detector, a technique that was subsequently used in the
experiment. From 1989 to 1994 he worked in the characterization of
various Tetramethyl liquids as part of a research project to find
appropriate warm liquids media for the envisioned calorimeter
detectors at SSC. He also worked in the construction, installation and
calibration of the Central Muon Extension (CMX) system for the CDF
detector. From 1994 to 1996 he developed an algorithm to improve
electron identification for the CDF end-plug ECAL based on the
information from the calorimeter and hits on the silicon tracker
detector. The algorithm led to the development and implementation of
the PHOENIX tracker system in the CDF-II detector. 
In the period of 2000–2003, he was the coordinator of the group
responsible for the development, installation, maintenance and
performance monitoring of the CDF-II Hadronic Calorimeter (HCAL)
timing system. For the entire period of the Tevatron Run-II
(2001-2011) he served as the coordinator of the CDF central HCAL
calibration (CHA and WHA), maintenance and performance group. Since
2004, when he joined the faculty of the UCY Physics Department, he has
been involved in the UCY HEP group activities related to the
construction and running of the CMS ECAL at CERN. In 2009, he
initiated the involvement of the group in the activities related to
the CMS tracking detector. He was also involved in the development of
the dual-readout calorimetry concept in a total absorption HCAL for
future linear-collider experiments. 
Professor Ptochos has led numerous physics analyses, spanning
from precision measurements on properties of heavy flavour quark
production and their use as probes for searching for the SM and SUSY
Higgs bosons, to searches for BSM physics including SUSY, extra
dimensions and other exotic processes. He has tremendous experience in
heavy flavour tagging techniques and algorithms, tau-lepton
identification techniques and new physics model building. 
He was co-coordinator of the research program
\say{High-Energy Physics with the CMS Experiment at LHC, CERN} that
received funding by the Cyprus RPF (2007-2012). He was also the
coordinator of the research project \say{Search for light neutral NMSSM
Higgs bosons at CMS}, also funded by the Cyprus RPF (2009- 2012) and
co-coordinator of a European Regional Development Fund with title \say{The
Regional Europe at the centre of the modern scientific research - A
Collaboration between Greece and Cyprus in High-Energy Physics and
Cosmology} (2006-2008). He was also the coordinator of a three-year
research project on \say{Search for neutral SM and MSSM Higgs bosons in
  the decay channel $H^{0}/A^{0}/h^{0} \to \tau^{+}\tau^{-}$}”
funded by the Cyprus RPF (2011-2014). He was also the first
ever recipient of the \say{Fermi National Accelerator Laboratory Fellowship}
for international senior researches (2007-2008). Soon after he was
the Scientist in Charge of a Marie-Curie Intra-European Fellowship
(IEF) entitled \say{HLTaus: A Level-1 Tau Trigger for CMS at HL-LHC}
(2014-2016) and of a two year project funded by UCY on \say{Tau hadronic
triggers in the HL-LHC era} (2016-2018). Most recently he coordinated
 two research projects funded by the Cyprus RPF on \say{Search
for a charged Higgs boson in a neutral Higgs boson and a W boson
final state} and \say{Search for charged Higgs bosons decaying to W and
neutral Higgs bosons with deep learning techniques} (2019-2021). He
is also a member of the European COST action CA16108 on \say{VBS scan (Vector
Boson Scattering)} (2018 – 2021). 
Professor Ptochos has been the author and co-author of more than 
1700 publications in refereed scientific journals and a member of the
editorial group in charge for producing the education material
(student instruction books and corresponding laboratory activities,
teachers’ instruction manuals) for the entire Cyprus Secondary
Education. In addition, he has been the supervisor of the research
activities of six postdoctoral fellows, five PhD and eleven MSc
students, as well as the theses projects of more than 20 undergraduate students.

    % ========================
    \begin{figure}
      \begin{center}
        \vspace{-0.2in}
        \leavevmode
        \includegraphics[width=0.5\textwidth]{./figures/Fotis6.png}
      \end{center}
    \end{figure}
    % ========================
  \end{multicols}
\end{headline}
