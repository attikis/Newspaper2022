%%%%%%%%%%%%%%%%%%%%%%%%%%%%%%%%%%%%%%%%%%%%%%%%%%%%%%%%%%%%%%%%%%%%%%
%% LaTeX (Beamer) Document for Presentations 
%% File ..............: ColumnOne.tex
%% Author ............: Alexandros X. Attikis
%% Institute .........: University of Cyprus (UCY)
%% e-Mail ............: attikis@cern.ch
%% Comments ..........: 
%%%%%%%%%%%%%%%%%%%%%%%%%%%%%%%%%%%%%%%%%%%%%%%%%%%%%%%%%%%%%%%%%%%%%%

%%%%%%%%%%%%%%%%%%%%%%%%%%%%%%%%%%%%%%%%%%%%%%%%%%%%%%%%%%%%%%%%%%%%%%
%%% Block 1
%%%%%%%%%%%%%%%%%%%%%%%%%%%%%%%%%%%%%%%%%%%%%%%%%%%%%%%%%%%%%%%%%%%%%%
\begin{MyColumnLeft}[detach title,before upper={\tcbtitle\quad}]{Τι είναι το μιόνιο$\bf{;}$}{34.0cm}

  Το μιόνιο αποτελεί ένα από τα 12 στοιχειώδη σωματίδια
  του Καθιερωμένου Πρωτύπου. Ανακαλύφθηκε το 1936 από τους
  \en Anderson \gr και \en Neddermeyer \gr με ένα θάλαμο νέφωσης
  εκτεθειμένο σε κοσμικές ακτίνες. Ανήκει στην οικογένεια των
  λεπτονίων και είναι μία βαρύτερη και ασταθής παραλλαγή του
  ηλεκτρονίου. Έχει αρνητικό ηλεκτρικό φορτίο, ιδιοστροφορμή
  $\nicefrac{1}{2}$, και μάζα 207 φορές μεγαλύτερη από αυτήν του ηλεκτρονίου.
 
  %%%%%%%%%%%%%%%%%%%%%%%%%%%%%%
  \renewcommand{\arraystretch}{1.15}
  \begin{table}[p]
    \caption{Ιδιότητες των μιονίων και ηλεκτρονίων.} 
    %\label{tab:emu-properties}
    \begin{center}
      \begin{tabular}{ l c c c}
        \hline
        & Μιόνιο (\lMu{}) & Ηλεκτρόνιο (\lE{}) & Μονάδες \\
        \hline      
        % Αναλλοίωτη Μάζα    & $106$              & $0.511$            & \en\MeVc{-2}\\
        Αναλλοίωτη Μάζα    & $1.9\times10^{-28}$ & $9.1\times10^{-31}$ & \en\sKiloGram\\
        Μέσος Χρόνος Ζωής  & $2.20\times10^{-6}$ & $\infty$           & \en\sSecond\\
        Ηλεκτρικό Φορτίο   & $-1$               & $-1$               & \lE{}\\
        Ιδιοστροφορμή      & $\nicefrac{1}{2}$  & $\nicefrac{1}{2}$  & $\hslash$\\
        \hline
      \end{tabular}
    \end{center}
  \end{table}
  \renewcommand{\arraystretch}{1.0}
  \vspace{0.5cm}
  %%%%%%%%%%%%%%%%%%%%%%%%%%%%%%
  Τα μιόνια είναι ασταθή σωματίδια με μέση διάρκεια ζωής $2.2 \times 10^{-6}$ δευτερόλεπτα (2.2 \,\sMicroSecond). Η διάσπαση τους στο κενό γίνεται κατά κύριο
  λόγο μέσω της Ασθενούς Αλληλεπίδρασης σε ένα ηλεκτρόνιο και δύο νετρίνο, με την αντίδραση $\lMu{-} \to \lE{-} \lANu{\lE{}} \lNu{\lMu{}}$.
  %%%%%%%%%%%%%%%%
  \oneFigPoster
  {muon-decay}
  {muon_decay.pdf}
  {0.50}
  {Διάσπαση του μιονίου σε ένα ηλεκτρόνιο και δύο νετρίνο.}
  %%%%%%%%%%%%%%%%

\end{MyColumnLeft}
%%%%%%%%%%%%%%%%%%%%%%%%%%%%%%%%%%%%%%%%%%%%%%%%%%%%%%%%%%%%%%%%%%%%%%


%%%%%%%%%%%%%%%%%%%%%%%%%%%%%%%%%%%%%%%%%%%%%%%%%%%%%%%%%%%%%%%%%%%%%%
%%% Block 2
%%%%%%%%%%%%%%%%%%%%%%%%%%%%%%%%%%%%%%%%%%%%%%%%%%%%%%%%%%%%%%%%%%%%%%
\begin{MyColumnLeft}[detach title,before upper={\tcbtitle\quad}]{Τι είναι η κοσμική ακτινοβολία$\bf{;}$}{34.0cm}

Ο αριθμός των μιονίων που παρατηρείται στο επίπεδο της θάλασσας προέρχεται κατά
κύριο λόγο από τις κοσμικές ακτίνες. Η κοσμική ακτινοβολία
περιλαμβάνει ηλεκτρικά φορτισμένα σωματίδια και πυρήνες υψηλών
ενεργειών που βομβαρδίζουν συνεχώς τα υψηλότερα στρώματα της ατμόσφαιρας.
Η ένταση αυτής της ακτινοβολίας αυξάνεται με το υψόμετρο, απόδειξη ότι προέρχεται
από το διάστημα. \\

Κατά συντριπτικό ποσοστό η ακτινοβολία αποτελείται από πρωτόνια
και πυρήνες ηλίου και προέρχεται από φαινόμενα εντός και εκτός του
γαλαξία μας, όπως πυρηνικές αντιδράσεις στον ήλιο, μαύρες τρύπες,
περιστρεφόμενα αστέρια νετρονίου και εκρήξεις υπερκαινοφανών αστέρων (σουπερνόβα). 

%%%%%%%%%%%%%%%%%%%%%%%%%%%%%%
\renewcommand{\arraystretch}{1.15}
\begin{table}[p]
  \caption{Σύνθεση της κοσμικής ακτινοβολίας.}%~\cite{PDG}
  \centering
  \begin{tabular}{l c}
    \hline
    Πηγή & Ποσοστό ($\%$) \\
    \hline
    Πρωτόνια      & $90$ \\
    Πυρήνες ηλίου & $8$ \\
    Ηλεκτρόνια    & $1$ \\
    Άλλα (λίθιο, βηρύλλιο, βόριο, \ldots) & $ <1$ \\
    \hline
  \end{tabular}
\end{table}
\renewcommand{\arraystretch}{1.0}
\vspace{0.5cm}
%%%%%%%%%%%%%%%%%%%%%%%%%%%%%%

Η αλληλεπίδραση των κοσμικών ακτίνων με τα μόρια του αέρα στα ψηλά
στρώματα της ατμόσφαιρας της Γης, παράγει δευτερεύουσα ακτινοβολία που
αποτελείται από πρωτόνια ($p$), νετρόνια ($n$), πιόνια (\mPion{0},
\mPion{\pm}), καόνια (\mKaon{0}{}, \mKaon{\pm}{}), ηλεκτρόνια (\lE{\pm})
και φωτόνια. Αυτά τα σωματίδια αλληλεπιδρούν μέσω Ηλεκτρομαγνητικών και Ασθενών
Δυνάμεων με πυρήνες των ατόμων της ατμόσφαιρας, σε μία συνεχιζόμενη διαδικασία που παίρνει
την μορφή καταιγίδας σωματιδίων.

\end{MyColumnLeft}
%%%%%%%%%%%%%%%%%%%%%%%%%%%%%%%%%%%%%%%%%%%%%%%%%%%%%%%%%%%%%%%%%%%%%%


%%%%%%%%%%%%%%%%%%%%%%%%%%%%%%%%%%%%%%%%%%%%%%%%%%%%%%%%%%%%%%%%%%%%%%
%%% Block 3
%%%%%%%%%%%%%%%%%%%%%%%%%%%%%%%%%%%%%%%%%%%%%%%%%%%%%%%%%%%%%%%%%%%%%%
\begin{MyColumnLeft}[detach title,before upper={\tcbtitle\quad}]{Τι είναι το κοσμικό μιόνιο$;$}{34.0cm}

Αυτή η καταιγίδα σωματιδίων, καθώς ταξιδεύει μέσα στην ατμόσφαιρα
της Γης, μπορεί όχι μόνο να φτάσει στην επιφάνεια της αλλά και να τη
διαπεράσει. Στο επίπεδο της θάλασσας η καταιγίδα αυτή περιέχει
κυρίως μιόνια. Τα μιόνια αυτά, που τα χαρακτηρίζουμε ως κοσμικά, παράγονται
ψηλά στην ατμόσφαιρα, τυπικά στα 15 χιλιόμετρα περίπου, μέσω των διασπάσεων $\mPion{-}
\to \lMu{-}\lANu{\lMu{}}$ και $\mPion{+} \to \lMu{+}\lNu{\lMu{}}$.\\

Τα κοσμικά μιόνια ταξιδεύουν με ταχύτητα κοντά στη ταχύτητα του φωτός
($\sVelocity{}\simeq \sSpeedOfLight$), διαπερνούν την ατμόσφαιρα 
και αλληλεπιδρούν με τα ηλεκτρικά πεδία των μορίων του αέρα ιονίζοντας
τα. Η απώλεια ενέργειας είναι σχετικά μικρή και έτσι μπορούν
να φτάσουν στην επιφάνεια της Γης. Η μέση ροή των κοσμικών μιονίων
είναι 1 μιόνιο ανά λεπτό ανά τετραγωνικό εκατοστό (1 \en\sMinuteCentimeter{-1}{-2}\gr).

%%%%%%%%%%%%%%%%
\twoFigPoster
{cosmic-shower}
{cosmic_shower.png}{1.0}
{cosmic_flux_pdg.pdf}{0.85}%{cosmic_rays_cropped}
{Δημιουργία καταιγίδας σωματιδίων στην ατμόσφαιρα (αριστερά) και
  κάθετες ροές κοσμικών ακτίνων στην ατμόσφαιρα με ενέργεια $\en> 1\GeV\gr$ (δεξιά).}
%%%%%%%%%%%%%%%%

\end{MyColumnLeft}
%%%%%%%%%%%%%%%%%%%%%%%%%%%%%%%%%%%%%%%%%%%%%%%%%%%%%%%%%%%%%%%%%%%%%%

%%% Vertical Fluxes Picture (Particle Data Group)
% Figure 24.3: Vertical fluxes of cosmic rays in the atmosphere with E
% > 1 GeV estimated from the nucleon flux of Eq. (24.2). The points
% show measurements of negative muons with Eμ > 1 GeV [32–36]. 
