%%%%%%%%%%%%%%%%%%%%%%%%%%%%%%%%%%%%%%%%%%%%%%%%%%%%%%%%%%%%%%%%%%%%%%
%% LaTeX (Beamer) Document for Presentations 
%% File ..............: ColumnTwo.tex
%% Author ............: Alexandros X. Attikis
%% Institute .........: University of Cyprus (UCY)
%% e-Mail ............: attikis@cern.ch
%% Comments ..........: 
%%%%%%%%%%%%%%%%%%%%%%%%%%%%%%%%%%%%%%%%%%%%%%%%%%%%%%%%%%%%%%%%%%%%%%

%%%%%%%%%%%%%%%%%%%%%%%%%%%%%%%%%%%%%%%%%%%%%%%%%%%%%%%%%%%%%%%%%%%%%%
%%% Block 4
%%%%%%%%%%%%%%%%%%%%%%%%%%%%%%%%%%%%%%%%%%%%%%%%%%%%%%%%%%%%%%%%%%%%%%
\begin{MyColumnCenter}[detach title,before upper={\tcbtitle\quad}]{Τι είναι ο θάλαμος νέφωσης$\bf{;}$}{51.5cm}

  Ο θάλαμος νέφωσης είναι μία συσκευή που κάνει ορατές τις τροχιές
  ηλεκτρικά φορτισμένων σωματιδίων όταν την διαπερνούν. Όταν ένα τέτοιο
  σωματίδιο εισέρχεται στον θάλαμο νέφωσης ιονίζει τα 
  μόρια του αέρα, δημιουργώντας λεπτές τροχιές υγροποίησης, λόγω
  συμπύκνωσης ατόμων υπέρκορης αλκοόλης. Αυτό το φαινόμενο είναι
  παρόμοιο με τη δημιουργία τροχιών συμπύκνωσης στον ουρανό πίσω από αεροπλάνα.\\
  %%%%%%%%%%%%%%%%
  \twoFigPoster
  {airplane-tracks}
  {cloud_chamber_picture_cropped}{1.0}
  {airplane_tracks}{1.0}
  {Τροχιές συμπκύκνωσης από κοσμικές ακτίνες σε θάλαμο νέφωσης (αριστερά) και τροχιές συμπύκνωσης από αεροπλάνα στον αέρα (δεξιά).}
  %%%%%%%%%%%%%%%%

  Ο θάλαμος νέφωσης αποτελείται από ένα διάφανο κουτί το
  οποίο κάθεται πάνω σε ένα στρώμα ξηρού πάγου (διοξείδιο του άνθρακα
  σε στερεά μορφή) που βρίσκεται σε θερμοκρασία $\en-78.5\sDegrees\mathrm{C}\gr$.
  Αυτό δημιουργεί μία βαθμίδα θερμοκρασίας μεταξύ της
  κορυφής του θαλάμου που βρίσκεται σε θερμοκρασία δωματίου και το κάτω μέρος
  στους $\en-78.5\sDegrees\mathrm{C}\gr$.\\

  Το απορροφητικό ύφασμα στο πάνω μέρος του θαλάμου είναι εμποτισμένο
  με αλκοόλ (αιθανόλη). Η αιθανόλη βρίσκεται σε
  θερμοκρασία δωματίου και εξατμίζεται δημιουργώντας ατμούς. 
  Οι ατμοί αιθανόλης ψύχονται από το κάτω μέρος του θαλάμου και αρχίζουν να
  συμπυκνώνονται σχηματίζοντας ένα σύννεφο από σταγονίδια αλκοόλης που
  πέφτουν προς τον πυθμένα. Ακριβώς πάνω από το σημείο όπου τα
  σταγονίδια σχηματίζονται υπάρχει μία περιοχή υπέρκορης
  αιθανόλης. Αυτό σημαίνει πως η θερμοκρασία 
  είναι αρκετά χαμηλή για την ύπαρξη σταγονιδίων, αλλά όχι αρκετά
  για την αυθόρμητη δημιουργία τους. Αυτή η περιοχή είναι πολύ
  ευαίσθητη σε οτιδήποτε που θα ευνοούσε τη δημιουργία σταγονιδίων
  αιθανόλης. \\

  Όταν ένα ηλεκτρικά φορτισμένο σωματίδιο υψηλής ενέργειας περνά πάνω
  από το υπέρκορο στρώμα αιθανόλης συγκρούεται διαδοχικά με μόρια
  αέρα, αφαιρώντας ηλεκτρόνια και αφήνοντας πίσω του ιόντα.
  Αυτά δημιουργούν ιδανικές συνθήκες για τη συμπύκνωση των ατμών
  αιθανόλης που υγροποιείται αφήνοντας πίσω της ένα ορατό ίχνος.
  Έτσι, είναι δυνατή η ανίχνευση του σωματιδίου μέσω της
  οπτικοποίησης της τροχιάς του!\\

%    Ο θάλαμος νέφωσης αποτελείται από αιθανόλη υγρής μορφής που
%  τοποθετείται μέσα σε ένα γυάλινο κουτί. Το κουτί κάθεται πάνω σε
%  ένα στρώμα ξηρού πάγου (διοξείδιο του άνθρακα σε στερεά μορφή) που βρίσκεται σε θερμοκρασία
%  -78.5$\sDegrees \mathrm{C}$, και η αιθανόλη ψύχεται μέχρι το σημείο
%  υπερκορεσμού της. \\
%
%  Οι ατμοί της αιθανόλης, που βρίσκονται στο πάνω
%  μέρος του θαλάμου, καθώς ψύχονται κατεβαίνουν προς τον πυθμένα του
%  θαλάμου. Εκεί, λόγω της χαμηλής θερμοκρασίας αρχίζουν να
%  υγροποιούνται, δημιουργώντας έτσι ένα στρώμα υπέρκορης ατμόσφαιρας
%  αιθανόλης πάνω από το στρώμα της υγρής αιθανόλης.\\

\end{MyColumnCenter}
%%%%%%%%%%%%%%%%%%%%%%%%%%%%%%%%%%%%%%%%%%%%%%%%%%%%%%%%%%%%%%%%%%%%%%


%%%%%%%%%%%%%%%%%%%%%%%%%%%%%%%%%%%%%%%%%%%%%%%%%%%%%%%%%%%%%%%%%%%%%%
%%% Block 5
%%%%%%%%%%%%%%%%%%%%%%%%%%%%%%%%%%%%%%%%%%%%%%%%%%%%%%%%%%%%%%%%%%%%%%
%\begin{MyColumnCenter}[detach title,before upper={\tcbtitle\quad}]{Τι βλέπω στο θάλαμο νέφωσης$;$ [1]}
\begin{MyColumnCenter}[detach title,before upper={\tcbtitle\quad}]{Τι βλέπω στο θάλαμο νέφωσης$;$}{51.5cm}

Οι τροχιές που παρατηρούμε μέσα στο θάλαμο νέφωσης σχηματίζονται 
κατά μήκος των διαδρομών κοσμικών ακτίνων. Οι χαρακτηριστικές τροχιές που παρατηρεί κάποιος είναι οι ακόλουθες:\\

\circled{1}{\MyHeadingsColor} Μακριά, λεπτή και σχετικά ευθεία
τροχία. \\Συνήθως προέρχεται από κοσμικό μιόνιο υψηλής ενέργειας.
%%%%%%%%%%%%%%%%
\oneFigPosterNoCaption
    {muon-track}
    {Chamber_Muon.pdf}
    {0.35}
%%%%%%%%%%%%%%%%

\circled{2}{\MyHeadingsColor} Κοντή, φαρδιά και έντονη ευθεία
τροχία. \\Προέρχεται από σωματίδιο άλφα που προκαλεί εκτεταμένο ιονισμό.
%%%%%%%%%%%%%%%%
\oneFigPosterNoCaption
    {alpha-track}
    {Chamber_Alpha.pdf}
    {0.28}
%%%%%%%%%%%%%%%%

\circled{3}{\MyHeadingsColor} Μακριά, λεπτή, ευθεία τροχία που αλλάζει απότομα διεύθυνση.\\
Αυτή είναι μία διάσπαση μιονίου σε ηλεκτρόνιο και νετρίνο $\lMu{-} \to \lE{-} \lANu{\lE{}} \lNu{\lMu{}}$.
Δεδομένου πως μόνο φορτισμένα σωματίδια μπορούμε να παρατηρήσουμε, τα
νετρίνο περνούν απαρατήρητα.
%%%%%%%%%%%%%%%%
\oneFigPosterNoCaption
    {muon-decay-track}
    {Chamber_MuonDecay.pdf}
    {0.35}
%%%%%%%%%%%%%%%%

\circled{4}{\MyHeadingsColor} Τρεις τροχιές που συναντιόνται σε ένα σημείο.\\
Η εισερχόμενη τροχιά είναι ένα κοσμικό μιόνιο που κτυπά ένα ατομικό
ηλεκτρόνιο και το ελευθερώνει. Τα προϊόντα της σκέδασης αυτής είναι το εκπεμπόμενο
ηλεκτρόνιο και το εκτρεπόμενο κοσμικό μιόνιο.
%%%%%%%%%%%%%%%%
\oneFigPosterNoCaption
    {electron-ionisation-track}
    {Chamber_ElectronIonisation.pdf}
    {0.35}
%%%%%%%%%%%%%%%%

\circled{5}{\MyHeadingsColor} Μία τροχιά ζιγκ-ζαγκ.\\
Αυτή είναι μία κοσμική ακτίνα χαμηλής ενέργειας που συγκρούεται
πολλαπλές φορές με διάφορα άτομα στον ενεργό όγκο του ανιχνευτή. %στον αέρα
%%%%%%%%%%%%%%%%
\oneFigPosterNoCaption
    {multiple-scattering-track}
    {Chamber_MultipleScatter.pdf}
    {0.40}
%%%%%%%%%%%%%%%%

\end{MyColumnCenter}
%%%%%%%%%%%%%%%%%%%%%%%%%%%%%%%%%%%%%%%%%%%%%%%%%%%%%%%%%%%%%%%%%%%%%%






%%%%%%%%%%%%%%%%%%%%%%%%%%%%%%%%%%%%%%%%%%%%%%%%%%%%%%%%%%%%%%%%%%%%%%
%%% http://techtv.mit.edu/videos/3141-cloud-chamber
%%%%%%%%%%%%%%%%%%%%%%%%%%%%%%%%%%%%%%%%%%%%%%%%%%%%%%%%%%%%%%%%%%%%%%
\begin{comment}
  concentrated rubbing alcohol is poured into a glass box that sits on a
  bed of dry ice. The alcohol becomes cooled to the point of
  supersaturation. when a high energy particle passes above the
  supersaturated layer it ionizes the air, causing the alcohol vapor to
  condense and leave a visible trail. subatomic particles such as cosmic
  rays muons, alpha particles and high energy electrons are  striking
  our bodies all the time. the cloud trails are formed along the paths
  of these particles. these cloud trails are about as close you will
  ever get to seeing subatomic particles with your own eyes!
  Subatomic particles such as cosmic ray muons, alpha particles, and
  high energy electrons are striking our bodies all the time. In the
  cloud chamber, these particles ionize air molecules, creating
  delicate cloud trails by condensing supersaturated alcohol
  vapor. This is similar to the way condensation trails are formed in
  the sky behind airplanes.  


  συμπυκνωμένο τρίψιμο αλκοόλη χύνεται σε ένα ποτήρι κουτί που κάθεται
  σε ένα κρεβάτι του ξηρού πάγου. Η αλκοόλη γίνεται ψύχεται μέχρι το
  σημείο υπερκορεσμού . όταν ένα υψηλής ενέργειας σωματιδίων περνά πάνω
  από το υπερκορεσμένο στρώμα που ιονίζει τον αέρα , προκαλώντας των
  ατμών της αλκοόλης να συμπυκνωθεί και να αφήσει ένα ορατό ίχνος . Τα
  υποατομικά σωματίδια όπως τα μιόνια των κοσμικών ακτίνων , τα
  σωματίδια άλφα και υψηλής ενέργειας ηλεκτρόνια εντυπωσιακό σώμα μας
  όλη την ώρα . τα μονοπάτια νέφος που σχηματίζεται κατά μήκος των
  διαδρομών αυτών των σωματιδίων . Αυτά τα μονοπάτια σύννεφο είναι
  περίπου τόσο κοντά θα πάρετε ποτέ να δούμε τα υποατομικά σωματίδια με
  τα δικά σας μάτια !
  Τα υποατομικά σωματίδια όπως τα μιόνια των κοσμικών ακτίνων , τα
  σωματίδια άλφα , και υψηλής ενέργειας ηλεκτρόνια εντυπωσιακό σώμα μας
  όλη την ώρα . Στο θάλαμο νέφους , αυτά τα σωματίδια ιονίζουν τα μόρια
  του αέρα , δημιουργώντας μονοπάτια λεπτό νέφος από συμπύκνωση
  υπερκορεσμένων ατμών αλκοόλης . Αυτό είναι παρόμοιο με τον τρόπο
  μονοπάτια συμπύκνωση σχηματίζονται στον ουρανό πίσω αεροπλάνα.
\end{comment}
%%%%%%%%%%%%%%%%%%%%%%%%%%%%%%%%%%%%%%%%%%%%%%%%%%%%%%%%%%%%%%%%%%%%%%

%%%%%%%%%%%%%%%%%%%%%%%%%%%%%%%%%%%%%%%%%%%%%%%%%%%%%%%%%%%%%%%%%%%%%%
% http://www.thenakedscientists.com/HTML/experiments/exp/cloud-chamber/
%%%%%%%%%%%%%%%%%%%%%%%%%%%%%%%%%%%%%%%%%%%%%%%%%%%%%%%%%%%%%%%%%%%%%%
\begin{comment}
  A piece of cloth soaked in alcohol (in this case iso-propanol), at
  room temperature the alcohol evaporates producing vapour. The bottom
  of the chamber is however very cold, this means that as the vapour
  falls down it cools until it eventually starts to condense, forming
  a cloud of alcohol droplets which fall down to the plate at the
  bottom. 

  Just above the point where the droplets are forming naturally there is
  an area of supersaturated vapour. This means that it is cool enough
  for droplets to grow, but not cool enough for the droplets to form
  spontaneously. This region is very sensitive to anything which might
  make forming a droplet slightly easier, so as soon as a droplet is
  started it will grow and grow.
  
  If a charged particle of radiation moves through the air, it will tend
  to crash into a series of air molecules leaving a trail of charged
  ions. This can then form the nuclei of droplets which then grow. This
  means that in the wake of the charged ions a line of droplets - a
  cloud trail - is formed allowing you to see their paths.
  \end{comment}
%%%%%%%%%%%%%%%%%%%%%%%%%%%%%%%%%%%%%%%%%%%%%%%%%%%%%%%%%%%%%%%%%%%%%%
